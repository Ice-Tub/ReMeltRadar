% -----------------------------------------------------------------------------
% Created:      2022-01-27
% Description:  Write-up of JH notebook and file log
%
% Requires:     csvsimple 
% -----------------------------------------------------------------------------

\documentclass[a4paper]{article}


% csvsimple package is required to load text from CSV files into tabular form
\usepackage{csvsimple}
\usepackage{xstring}
\usepackage{longtable}
\usepackage[table]{xcolor}
\usepackage{hyperref}

% array package for column modifications (emphasis)
\usepackage{array}

% for convering datetimes
\usepackage[british,showdow]{datetime2}

\usepackage{tikz}

\newcommand\hfaprestable[9]{
    % Arguments
    %   1 - filename
    %   2 - number of sub bursts
    %   3 - number of attenuator settings
    %   4 - AF gain (list)
    %   5 - RF attenuator (list)
    %   6 - lower frequency
    %   7 - upper frequency
    %   8 - chirp period
    %   9 - location
    %  10 - comments
    \def\temphftfolder{#1}
    \def\temphftfilename{#2}%
    \def\temphftsubbursts{#3}%
    \def\temphftnoattn{#4}
    \def\temphftaf{#5}%
    \def\temphftrf{#6}%
    \def\temphftlowfreq{#7}%
    \def\temphftupfreq{#8}%
    \def\temphftchirp{#9}%
    \hfaprestablecont
}

\newcommand{\apresdoc}[9]{
    \def\tempapfname{#1}
    \def\tempaploc{#2}
    \def\tempapcom{#3}
    \def\temptimestamp{#4}
    \def\tempafgain{#5}
    \def\temprfattn{#6}
    \def\tempperiod{#7}
    \def\tempflow{#8}
    \def\tempfupp{#9}
    \apresdoccont
}
\newcommand{\apresdoccont}[5]{
    \def\tempnatt{#1}
    \def\tempnchirp{#2}
    \def\tempnsubburst{#3}
    \def\temppowercode{#4}
    \def\tempbattvolt{#5}

    \renewcommand{\arraystretch}{1.5}
    \rowcolors{2}{gray!10}{white}
    \begin{longtable}{>{\small\bfseries\raggedright}p{0.25\textwidth} >{\small\raggedright\arraybackslash}p{0.75\textwidth}}
        \hline
        \multicolumn{2}{c}{\bfseries Filename: \tempapfname} \\
        \hline
        \multicolumn{2}{c}{\includegraphics[width=\textwidth]{Testing/figures/\tempapfname.png}} \\
        \hline
        Time & \temptimestamp \\
        Location & \tempaploc \\
        Comments & \tempapcom \\
        AF Gain & \tempafgain \\
        RF Attenuation & \temprfattn \\
        Period & \tempperiod~s \\
        Bandwidth & \tempflow~Hz to \tempfupp~Hz \\
        N. Attenuators & \tempnatt \\
        N. Sub Bursts & \tempnsubburst \\
        Power Code & \temppowercode \\
        Battery Voltage & \tempbattvolt~V \\
        \hline
    \end{longtable}

    % \subsubsection*{\tempapfname}
    % \includegraphics[width=\textwidth]{Testing/figures/\tempapfname.png}
    % \paragraph{Location:} \tempaploc
    % \paragraph{Comments:} \tempapcom
    \newpage
}

% -----------------------------------------------------------------------------
% Define command for database defintion tables
\newcommand{\sqlspectable}[2]{
    \renewcommand{\arraystretch}{1.5}
    \rowcolors{2}{gray!10}{white}
    \begin{longtable}{l l >{\raggedright}p{3cm} >{\raggedright\arraybackslash}p{4cm}}
        \caption{#2} \\
        \hline
        \textbf{Fieldname} & \textbf{Datatype} & \textbf{Parameters} & \textbf{Description}\\
        \hline
        \endhead
        \hline
        \endfoot
        #1
    \end{longtable}
}

% -----------------------------------------------------------------------------
% Define new date style for subsection titles

% -----------------------------------------------------------------------------
% Title
\title{\textbf{Notebook Summary for HF ApRES Experiments}}
\author{J.D.~Hawkins}

\begin{document}

    \maketitle
    
    \newpage

    \tableofcontents

    \newpage

    \section{Database Structure}

    \subsection{Table `\texttt{measurements}'} 
    \sqlspectable{
        %
        measurement\_id & INTEGER & PRIMARY KEY & 
        Unique identifier for each file. \\
        %
        filename & TEXT & NOT NULL & 
        Filename without path. \\
        %
        path & TEXT & UNIQUE NOT NULL&
        Path to file in UNIX form, relative to top-level project root. \\
        %
        name & TEXT & - &
        The name associated with the measurement, used to group sets of 
        measurements together. \\
        %
        timestamp & TEXT & UNIQUE ASC NOT NULL &
        \texttt{YYYY-mm-dd HH:MM:SS.fff} formatted timestamp according to
        time and date the measurement was taken. \\
        %
        valid & INTEGER & NOT NULL DEFAULT 0 &
        Boolean indicator of whether file is valid. Assumes invalid by 
        default. \\
        %
        base\_visible & INTEGER & NOT NULL DEFAULT 0 & 
        Boolean indicator of whether basal reflector is visible in range data. \\
        %
        base\_range\_min & REAL & NOT NULL DEFAULT -1 & 
        Minimum range (in steps of 50m) at which basal reflector can be found. \\
        %
        base\_range\_max & REAL & NOT NULL DEFAULT -1 & 
        Maximum range (in steps of 50m) at which basal reflector can be found. \\
        %
        location & TEXT & - & 
        Measurement location name. \\
        % 
        comments & TEXT & - &
        Description and comments for measurement if relevant. \\
        % 
        latitude & REAL & - &
        Latitude of measurement location if known. \\
        %
        longitude & REAL & - & 
        Longitude of measurement location if known. \\
        %
        elevation & REAL & - & 
        Elevation of measurement location if known (referenced to WGS84). \\
    }
    {Specification for \texttt{measurements} table.}
    
    \newpage

    \subsection{Table `\texttt{apres\_metadata}'}
    The fields \texttt{id} and \texttt{burst\_id} make a unique pair to ensure
    that each burst within a \texttt{*.dat} file is only represented once.

    \sqlspectable{
        %
        id & INTEGER & PRIMARY KEY & 
        Unique identifier for metadata.  Distinct from \texttt{measurement\_id}
        in that each \texttt{*.dat} file can have multiple bursts.\\
        %
        burst\_id & INTEGER & NOT NULL &
        Identifies the burst within a \texttt{*.dat} the metadata represents.\\
        %
        measurement\_id & INTEGER & NOT NULL & 
        Identifies the file record where the metadata originates from. \\
        %
        % header\_index & INTEGER & NOT NULL DEFAULT 0 &
        % Indicates the byte at which to start reading the header \\
        % % 
        % burst\_index & INTEGER & NOT NULL & 
        % Indicates the byte at which to start reading the data associated with the burst \\
        %
        \hline
        \multicolumn{4}{c}{\textit{ApRES Specific Metadata}} \\
        \hline
        %
        timestamp & TEXT & NOT NULL & 
        \texttt{YYYY-mm-dd HH:MM:SS.fff} formatted timestamp as logged in
        \texttt{*.dat} file. \\
        %
        n\_attenuators & INTEGER & NOT NULL CHECK(\textgreater 0, \textless 5) & 
        Number of attenuator settings used. \\
        %
        n\_chirps & INTEGER & NOT NULL CHECK(\textgreater 0) & 
        Total number of individual chirps in file. \\
        %
        n\_subbursts & INTEGER & NOT NULL CHECK(\textgreater 0) &
        Number of sub-bursts (repeats) of the burst configuration. \\
        %
        period & REAL & NOT NULL CHECK(\textgreater 0) & 
        Chirp period in seconds. \\
        %
        f\_lower & REAL & NOT NULL CHECK($\ge$ 0) &
        Lower bound of chirp ramp in Hertz. \\
        %
        f\_upper & REAL & NOT NULL CHECK($\ge$ 0) & 
        Upper bound of chirp ramp in Hertz. \\
        %
        af\_gain & TEXT & NOT NULL &
        Comma separated values indicating AF gain settings. \\
        %
        rf\_attenuator & TEXT & NOT NULL & 
        Comma separated values indicating RF attenuator settings. \\
        %
        f\_sampling & REAL & NOT NULL CHECK(\textgreater 0) & 
        Sampling frequency. \\
        % 
        tx\_antenna & TEXT & NOT NULL &
        Transmit antenna selection in comma separated value format. \\
        % 
        rx\_antenna & TEXT & NOT NULL &
        Receive antenna selection in comma separated value format. \\
        % 
        power\_code & INTEGER & - &
        DDS output current power code, if available (dependent on firmware).\\
        %
        battery\_voltage & REAL & - &
        Battery voltage, if available. \\
        %
        temperature\_1 & REAL & - &
        Measured board temperature from sensor 1. \\
        %
        temperature\_2 & REAL & - & 
        Measured board temperature from sensor 2. \\
        %
        rmb\_issue & TEXT & - & 
        RMB issue number if available. \\
        % 
        vab\_issue & TEXT & - & 
        VAB issue number if available. \\
        %
        venom\_issue & TEXT & - &
        Venom issue number if available. \\
        %
        software\_issue & TEXT & - & 
        VAB firmware issue number if available. \\
    }{
        Specification for \texttt{apres\_metadata} table.
    }
    
    \newpage

    \subsection{Table `\texttt{data}'}
    \sqlspectable{
        data\_id & INTEGER & PRIMARY KEY & 
        Unique identifier for each data item stored in \texttt{/Proc}. \\
        %
        measurement\_id & INTEGER & NOT NULL &
        Linked identified to \texttt{measurements} table for source of
        data. \\
        % 
        filename & TEXT & NOT NULL &
        Filename for processed data item stored in \texttt{/Proc}. \\
        %
        path & TEXT & NOT NULL & 
        Path to file in UNIX form, relative to top-level project root. \\
        %
        timestamp&  TEXT & ASC NOT NULL &
        \texttt{YYYY-mm-dd HH:MM:SS.SSS} formatted timestamp corresponding
        to time at which data was processed. \\
        %
        processing\_steps & TEXT & - & 
        Description of processing steps used to produced data file. \\
        %
    }
    {Specification for \texttt{data} table.}

    \newpage

    \section{Testing}

    \apresdoc{2021-12-27\_214306.dat}{Neumayer III,Storage Container}{Tested outside storage container.  Some clipping evident.  In general much higher frequency content that other signals.}{2021-12-27 21:43:11.000}{6,6,6,6}{30,20,10,0}{1.000}{20000000.000}{40000000.000}{4}{80}{20}{127}{12.194}
\apresdoc{hf-apres-test-2021-12-28.dat}{Neumayer III}{Antennas aligned E-W, no receive antenna connected.}{2021-12-28 21:29:22.000}{6,6,6,6}{30,20,10,0}{1.000}{20000000.000}{40000000.000}{4}{4}{1}{127}{12.133}
\apresdoc{2021-12-28\_213612.dat}{Neumayer III}{Antennas aligned E-W, no receive antenna connected.}{2021-12-28 21:36:13.000}{6,6,6,6}{30,20,10,0}{1.000}{20000000.000}{40000000.000}{4}{4}{1}{127}{12.153}
\apresdoc{2021-12-28\_213633.dat}{Neumayer III}{Antennas aligned E-W, no receive antenna connected.}{2021-12-28 21:36:34.000}{6,6,6,6}{30,20,10,0}{1.000}{20000000.000}{40000000.000}{4}{4}{1}{127}{12.129}
\apresdoc{2021-12-28\_213647.dat}{Neumayer III}{Antennas aligned E-W, no receive antenna connected.}{2021-12-28 21:36:48.000}{6,6,6,6}{30,20,10,0}{1.000}{20000000.000}{40000000.000}{4}{4}{1}{127}{12.117}
\apresdoc{2021-12-28\_213710.dat}{Neumayer III}{Antennas aligned E-W, no receive antenna connected.}{2021-12-28 21:37:10.000}{6,6,6,6}{30,20,10,0}{1.000}{20000000.000}{40000000.000}{4}{4}{1}{127}{12.113}
\apresdoc{2021-12-28\_213752.dat}{Neumayer III}{Antennas aligned E-W, no receive antenna connected.}{2021-12-28 21:37:53.000}{6,6,6,6}{30,20,10,0}{1.000}{20000000.000}{40000000.000}{4}{40}{10}{127}{12.129}
\apresdoc{2021-12-28\_214326.dat}{Neumayer III}{Antennas aligned E-W, connect Rx antenna (labelled \#1).}{2021-12-28 21:43:26.000}{6,6,6,6}{30,20,10,0}{1.000}{20000000.000}{40000000.000}{4}{40}{10}{127}{12.158}
\apresdoc{2021-12-28\_214449.dat}{Neumayer III}{Antennas aligned E-W.}{2021-12-28 21:44:49.000}{6}{30}{1.000}{20000000.000}{40000000.000}{1}{10}{10}{127}{12.117}
\apresdoc{2021-12-28\_214537.dat}{Neumayer III}{Antennas aligned E-W.}{2021-12-28 21:45:37.000}{-4}{30}{1.000}{20000000.000}{40000000.000}{1}{10}{10}{127}{12.129}
\apresdoc{2021-12-28\_214702.dat}{Neumayer III}{Antennas aligned E-W.}{2021-12-28 21:47:02.000}{-4}{30}{1.000}{20000000.000}{40000000.000}{1}{10}{10}{127}{12.149}
\apresdoc{2021-12-28\_220421.dat}{Neumayer III}{Antennas aligned E-W.}{2021-12-28 22:04:21.000}{-4}{1}{1.000}{20000000.000}{40000000.000}{1}{10}{10}{127}{12.250}
\apresdoc{2021-12-28\_220508.dat}{Neumayer III}{Antennas moved fruther from Neumayer, in line with radio mast.}{2021-12-28 22:05:08.000}{-4}{30}{1.000}{20000000.000}{40000000.000}{1}{10}{10}{127}{12.178}
\apresdoc{2021-12-28\_220547.dat}{Neumayer III}{Antennas moved fruther from Neumayer, in line with radio mast. Antennas disconnected.}{2021-12-28 22:05:47.000}{-14}{30}{1.000}{20000000.000}{40000000.000}{1}{10}{10}{127}{12.141}
\apresdoc{2021-12-28\_220726.dat}{Neumayer III}{Antennas moved fruther from Neumayer, in line with radio mast.  Antennas disconnected.}{2021-12-28 22:07:27.000}{-14}{30}{1.000}{20000000.000}{40000000.000}{1}{10}{10}{127}{12.162}
\apresdoc{2021-12-28\_221042.dat}{Neumayer III}{Antennas moved fruther from Neumayer, in line with radio mast.  Antennas disconnected.}{2021-12-28 22:10:43.000}{-14}{30}{1.000}{20000000.000}{40000000.000}{1}{1}{1}{64}{12.165}
\apresdoc{2021-12-28\_221153.dat}{Neumayer III}{Antennas moved fruther from Neumayer, in line with radio mast.  Antennas disconnected.}{2021-12-28 22:11:54.000}{-14}{0}{1.000}{20000000.000}{40000000.000}{1}{10}{10}{64}{12.170}
\apresdoc{2021-12-28\_221319.dat}{Neumayer III}{Antennas moved fruther from Neumayer, in line with radio mast.  Antennas disconnected.}{2021-12-28 22:13:19.000}{-14}{0}{1.000}{20000000.000}{40000000.000}{1}{10}{10}{64}{12.162}
\apresdoc{2021-12-28\_221436.dat}{Neumayer III}{Antennas moved fruther from Neumayer, in line with radio mast.  Antennas disconnected.}{2021-12-28 22:14:36.000}{-14}{0}{1.000}{20000000.000}{40000000.000}{1}{10}{10}{64}{12.158}
\apresdoc{2021-12-28\_221516.dat}{Neumayer III}{Antennas moved fruther from Neumayer, in line with radio mast.  Antennas disconnected.}{2021-12-28 22:15:17.000}{6}{30}{1.000}{20000000.000}{40000000.000}{1}{10}{10}{64}{12.141}
\apresdoc{2021-12-28\_221651.dat}{Neumayer III}{Antennas moved fruther from Neumayer, in line with radio mast.  Antennas disconnected.}{2021-12-28 22:16:51.000}{6}{30}{1.000}{20000000.000}{40000000.000}{1}{10}{10}{64}{12.162}
\apresdoc{2021-12-28\_221725.dat}{Neumayer III}{Antennas moved fruther from Neumayer, in line with radio mast.  Antennas disconnected.}{2021-12-28 22:17:25.000}{-4}{30}{1.000}{20000000.000}{40000000.000}{1}{10}{10}{64}{12.137}
\apresdoc{2021-12-28\_221834.dat}{Neumayer III}{Antennas moved fruther from Neumayer, in line with radio mast.  Antennas disconnected.}{2021-12-28 22:18:34.000}{-4}{30}{1.000}{20000000.000}{40000000.000}{1}{10}{10}{64}{12.153}
\apresdoc{2021-12-28\_221937.dat}{Neumayer III}{Antennas moved fruther from Neumayer, in line with radio mast.  Antennas disconnected.}{2021-12-28 22:19:37.000}{-14}{30}{1.000}{20000000.000}{40000000.000}{1}{1}{1}{64}{12.153}
\apresdoc{2021-12-28\_221958.dat}{Neumayer III}{Antennas moved fruther from Neumayer, in line with radio mast.  Antennas disconnected.}{2021-12-28 22:19:58.000}{6}{0}{1.000}{20000000.000}{40000000.000}{1}{1}{1}{64}{12.145}
\apresdoc{2021-12-28\_222040.dat}{Neumayer III}{Antennas moved fruther from Neumayer, in line with radio mast.  Antennas disconnected.  Note peak of ice shelf appearing at approx 250m.}{2021-12-28 22:20:40.000}{6}{0}{1.000}{20000000.000}{40000000.000}{1}{1}{1}{64}{12.158}
\apresdoc{2021-12-28\_222046.dat}{Neumayer III}{Antennas moved fruther from Neumayer, in line with radio mast.  Antennas disconnected.}{2021-12-28 22:20:47.000}{6}{0}{1.000}{20000000.000}{40000000.000}{1}{1}{1}{64}{12.145}
\apresdoc{2021-12-28\_222051.dat}{Neumayer III}{Antennas moved fruther from Neumayer, in line with radio mast.  Antennas disconnected.}{2021-12-28 22:20:51.000}{6}{0}{1.000}{20000000.000}{40000000.000}{1}{1}{1}{64}{12.133}
\apresdoc{2021-12-28\_223632.dat}{Neumayer III}{Antennas moved fruther from Neumayer, in line with radio mast.  Measurement after ApRES powered off for 10 minutes.}{2021-12-28 22:36:32.000}{-4}{30}{1.000}{20000000.000}{40000000.000}{1}{10}{10}{127}{12.278}
\apresdoc{2021-12-28\_223835.dat}{Neumayer III}{Antennas moved fruther from Neumayer, in line with radio mast.  Measurement after ApRES powered off for 10 minutes.}{2021-12-28 22:38:36.000}{-4}{30}{1.000}{20000000.000}{40000000.000}{1}{10}{10}{127}{12.162}
\apresdoc{2021-12-28\_224425.dat}{Neuayer III}{Change to 36-40 MHz chirp.  Ice shelf base clear but poor resolution.  Clipping with one AF/RF setting?}{2021-12-28 22:44:25.000}{-4}{30}{1.000}{36000000.000}{40000000.000}{1}{10}{10}{127}{12.137}
\apresdoc{2021-12-28\_224607.dat}{Neumayer III}{Additional chirp with 36-40 MHz bandwidth.  Clipping behaviour seen with constant AF/RF gain setting.}{2021-12-28 22:46:08.000}{-4}{30}{1.000}{36000000.000}{40000000.000}{1}{30}{30}{127}{12.129}
\apresdoc{2021-12-28\_225108.dat}{Neumayer III}{As before short bursts of intereference.  Ice-base still clear.}{2021-12-28 22:51:08.000}{-4}{30}{1.000}{36000000.000}{40000000.000}{1}{30}{30}{127}{12.137}
\apresdoc{2021-12-28\_233118.dat}{Neumayer III}{No comments noted.}{2021-12-28 23:31:18.000}{-14,-14,-14,-14}{30,20,10,0}{1.000}{20000000.000}{40000000.000}{4}{40}{10}{127}{12.162}
\apresdoc{2021-12-28\_233345.dat}{Neumayer III}{No comments noted.}{2021-12-28 23:33:46.000}{-14}{30}{1.000}{20000000.000}{40000000.000}{1}{1}{1}{127}{12.085}
\apresdoc{2021-12-28\_233417.dat}{Neumayer III}{No comments noted.}{2021-12-28 23:34:17.000}{-4}{30}{1.000}{20000000.000}{40000000.000}{1}{1}{1}{127}{12.097}
\apresdoc{2021-12-28\_233617.dat}{Neumayer III}{No comments noted.}{2021-12-28 23:36:18.000}{-4}{30}{1.000}{20000000.000}{40000000.000}{1}{1}{1}{127}{12.117}
\apresdoc{2021-12-28\_233652.dat}{Neumayer III}{No comments noted.}{2021-12-28 23:36:53.000}{-4}{30}{1.000}{20000000.000}{40000000.000}{1}{1}{1}{127}{12.117}
\apresdoc{2021-12-28\_233714.dat}{Neumayer III}{No comments noted.}{2021-12-28 23:37:14.000}{-4}{0}{1.000}{20000000.000}{40000000.000}{1}{1}{1}{127}{12.109}
\apresdoc{2021-12-28\_233729.dat}{Neumayer III}{No comments noted.}{2021-12-28 23:37:30.000}{-4}{10}{1.000}{20000000.000}{40000000.000}{1}{1}{1}{127}{12.109}
\apresdoc{2021-12-28\_233742.dat}{Neumayer III}{No comments noted.}{2021-12-28 23:37:43.000}{-4}{20}{1.000}{20000000.000}{40000000.000}{1}{1}{1}{127}{12.105}
\apresdoc{2021-12-28\_233844.dat}{Neumayer III}{No comments noted.}{2021-12-28 23:38:44.000}{-4}{30}{1.000}{20000000.000}{40000000.000}{1}{1}{1}{127}{12.101}
\apresdoc{2021-12-28\_233949.dat}{Neumayer III}{No comments noted.}{2021-12-28 23:39:49.000}{-4}{30}{1.000}{20000000.000}{40000000.000}{1}{1}{1}{127}{12.101}
\apresdoc{2021-12-28\_234104.dat}{Neumayer III}{No comments noted.}{2021-12-28 23:41:04.000}{-4}{30}{1.000}{20000000.000}{40000000.000}{1}{1}{1}{127}{12.109}
\apresdoc{2021-12-28\_234122.dat}{Neumayer III}{No comments noted.}{2021-12-28 23:41:23.000}{-4}{30}{1.000}{20000000.000}{40000000.000}{1}{1}{1}{127}{12.097}
\apresdoc{2021-12-28\_234222.dat}{Neumayer III}{No comments noted.}{2021-12-28 23:42:22.000}{-4}{30}{1.000}{20000000.000}{40000000.000}{1}{1}{1}{127}{12.101}
\apresdoc{2021-12-29\_130718.dat}{Neumayer III,500m W of Station}{Base visible but low power. Additional Rx 10dB, Tx 10dB attenuation.}{2021-12-29 13:07:19.000}{-14}{30}{1.000}{20000000.000}{40000000.000}{1}{1}{1}{127}{12.484}
\apresdoc{2021-12-29\_130920.dat}{Neumayer III,500m W of Station}{Base visible but low power. Increased AF gain but otherwise consistent with previous. Additional Rx 10dB, Tx 10dB attenuation.}{2021-12-29 13:09:21.000}{-4}{30}{1.000}{20000000.000}{40000000.000}{1}{1}{1}{127}{12.468}
\apresdoc{2021-12-29\_131140.dat}{Neumayer III,500m W of Station}{Base visible but low power. Increased AF gain again to 6dB.  High frequency transient at 0.075s. Additional Rx 10dB, Tx 10dB attenuation.}{2021-12-29 13:11:40.000}{6}{30}{1.000}{20000000.000}{40000000.000}{1}{1}{1}{127}{12.464}
\apresdoc{2021-12-29\_131353.dat}{Neumayer III,500m W of Station}{Base visible but low power. Increased AF gain again to 6dB.  Repeat of above, transient reduced. Additional Rx 10dB, Tx 10dB attenuation.}{2021-12-29 13:13:53.000}{6}{30}{1.000}{20000000.000}{40000000.000}{1}{1}{1}{127}{12.456}
\apresdoc{2021-12-29\_131431.dat}{Neumayer III,500m W of Station}{Keep AF gain at 6dB.  Increased noise floor more pronounced at 0dB RF attenuation.  Note initial transient (settling?). Additional Rx 10dB, Tx 10dB attenuation.}{2021-12-29 13:14:31.000}{6,6,6,6}{30,20,10,0}{1.000}{20000000.000}{40000000.000}{4}{4}{1}{127}{12.440}
\apresdoc{2021-12-29\_132044.dat}{Neumayer III,500m W of Station}{Significant clipping for all RF gain settings. Additional Rx 10dB.  Attenuator at Tx removed.}{2021-12-29 13:20:44.000}{6,6,6,6}{30,20,10,0}{1.000}{20000000.000}{40000000.000}{4}{4}{1}{127}{12.447}
\apresdoc{2021-12-29\_132217.dat}{Neumayer III,500m W of Station}{Reduced power code to 0. Output shows slight clipping near centre of chirp.}{2021-12-29 13:22:17.000}{6}{30}{1.000}{20000000.000}{40000000.000}{1}{1}{1}{127}{12.415}
\apresdoc{2021-12-29\_133008.dat}{Neumayer III,500m W of Station}{Reduced power code to 0. Output shows slight clipping near centre of chirp.}{2021-12-29 13:30:31.000}{6}{30}{1.000}{20000000.000}{40000000.000}{1}{1}{1}{0}{12.383}
\apresdoc{2021-12-29\_152201.dat}{Neumayer III,500m W of Station}{Measurements taken after 2hour break.  Cables sunk into snow.  Keep power output at 0.  Moved ApRES, battery  and operator (JH) into centre of ApRES.  Approx 3m from end of each antenna.}{2021-12-29 15:24:27.000}{6}{30}{1.000}{20000000.000}{40000000.000}{1}{1}{1}{0}{12.319}
\apresdoc{2021-12-29\_154120.dat}{Neumayer III,500m W of Station}{Move Tx and Rx antennas further 2m apart in each direction.  Total separation now 10m.  Increased clipping - perhaps Tx or Rx disconnected?}{2021-12-29 15:47:23.000}{6}{30}{1.000}{20000000.000}{40000000.000}{1}{1}{1}{0}{12.238}
\apresdoc{2021-12-29\_160719.dat}{Neumayer III,500m W of Station}{Add Tx attenuator (10dB).  Additional Rx attenuator 10dB.  Base reappears with double peak? }{2021-12-29 16:07:36.000}{6}{30}{1.000}{20000000.000}{40000000.000}{1}{10}{10}{0}{12.246}
\apresdoc{2021-12-29\_161210.dat}{Neumayer III,500m W of Station}{Additional Tx and Rx attenuator 10dB.  Base masked by clipping in low RF attenuation settings.}{2021-12-29 16:12:35.000}{6,6,6,6}{30,20,10,0}{1.000}{20000000.000}{40000000.000}{4}{4}{1}{0}{12.230}
\apresdoc{2021-12-29\_163501.dat}{Neumayer III,500m W of Station}{Additional Tx and Rx attenuator 10dB.  Change config to 10ms - base not visible.}{2021-12-29 16:37:31.000}{6,6,6,6}{30,20,10,0}{0.010}{20000000.000}{40000000.000}{4}{4}{1}{0}{12.266}
\apresdoc{2021-12-29\_164540.dat}{Neumayer III,500m W of Station}{Additional Tx and Rx attenuator 10dB.  Change config to 2s.  Clipping still evident in RF Attn=0.  Base visible in other settings - with some clipping and increased noise floor.}{2021-12-29 16:46:38.000}{6,6,6,6}{30,20,10,0}{2.000}{20000000.000}{40000000.000}{4}{4}{1}{0}{12.274}
\apresdoc{2021-12-29\_165611.dat}{Neumayer III,500m W of Station}{Additional Tx and Rx attenuator 10dB.  Reduce antenna sepraation to 6m.  Config sto;;2s.  Clipping still evident in RF Attn=0.  Otherwise improved.}{2021-12-29 16:56:35.000}{6,6,6,6}{30,20,10,0}{2.000}{20000000.000}{40000000.000}{4}{4}{1}{0}{12.278}
\apresdoc{2021-12-29\_170625.dat}{Neumayer III,500m W of Station}{Additional Tx and Rx attenuator 10dB.  Increase antenna sepraation to 14m.  Config Clipping evident in RF Attn=0dB, 10dB.  More similar to 10m separation than 6m.}{2021-12-29 17:06:36.000}{6,6,6,6}{30,20,10,0}{2.000}{20000000.000}{40000000.000}{4}{4}{1}{0}{12.286}
\apresdoc{2021-12-29\_171503.dat}{Neumayer III,500m W of Station}{Additional Tx and Rx attenuator 10dB.  Change orientation.  Tx (Ant1) on W of ApRES aligned N-S and Tx (Ant2) aligned EW as previous (i.e. Cross-pol).  Basal signal present and clipping for all except RF=0dB.}{2021-12-29 17:15:22.000}{6,6,6,6}{30,20,10,0}{2.000}{20000000.000}{40000000.000}{4}{4}{1}{0}{12.299}
\apresdoc{2021-12-29\_172626.dat}{Neumayer III,500m W of Station}{Additional Tx and Rx attenuator 10dB.  Change orientation of Rx (Ant2) aligned N-S as for Tx.  Basal signal present all near-field clutter reduced.  Strong basal signal.}{2021-12-29 17:26:30.000}{6,6,6,6}{30,20,10,0}{2.000}{20000000.000}{40000000.000}{4}{4}{1}{0}{12.306}
\apresdoc{2021-12-29\_173748.dat}{Neumayer III,500m W of Station}{Additional Tx and Rx attenuator 10dB.  Return antennas to E-W configuration.  Aligned both with +ve W, -ve E.  (Previously Tx was +E, -W and Rx +W -E).  Otherwise as above with 14m separation.}{2021-12-29 17:39:16.000}{6,6,6,6}{30,20,10,0}{2.000}{20000000.000}{40000000.000}{4}{4}{1}{0}{12.310}
\apresdoc{2021-12-29\_180059.dat}{Neumayer III,500m W of Station}{Additional Tx and Rx attenuator 10dB.  Rotate antennas so that they are E-W aligned but are broadside to each other.  Hence Tx now N and Rx S.  Separation is 14m.  One side of each dipole is unattached (presumably from dragging?).}{2021-12-29 18:02:00.000}{6,6,6,6}{30,20,10,0}{2.000}{20000000.000}{40000000.000}{4}{4}{1}{0}{12.310}
\apresdoc{2021-12-29\_180743.dat}{Neumayer III,500m W of Station}{Additional Tx and Rx attenuator 10dB.  Rotate antennas so that they are E-W aligned but are broadside to each other.  Hence Tx now N and Rx S.  Separation is 14m.  Reconnect dipole arms (previously unattached) - no signal now seen.}{2021-12-29 18:07:47.000}{6,6,6,6}{30,20,10,0}{2.000}{20000000.000}{40000000.000}{4}{4}{1}{0}{12.331}
\apresdoc{2021-12-29\_181349.dat}{Neumayer III,500m W of Station}{Additional Tx and Rx attenuator 10dB.  Rotate antennas so that they are E-W aligned but are broadside to each other.  Hence Tx now N and Rx S.  Separation is 14m.  Reconnect dipole arms (previously unattached) - no signal now seen with increased power code.}{2021-12-29 18:14:31.000}{6,6,6,6}{30,20,10,0}{2.000}{20000000.000}{40000000.000}{4}{4}{1}{127}{12.335}
\apresdoc{2021-12-29\_181757.dat}{Neumayer III,500m W of Station}{Additional Tx and Rx attenuator 10dB.  Rotate antennas so that they are E-W aligned but are broadside to each other.  Hence Tx now N and Rx S.  Separation is 14m.  Very low power basal return and high-frequency transient.}{2021-12-29 18:18:06.000}{6,6,6,6}{30,20,10,0}{2.000}{20000000.000}{40000000.000}{4}{4}{1}{127}{12.331}
\apresdoc{2021-12-29\_182318.dat}{Neumayer III,500m W of Station}{Additional Tx and Rx attenuator 10dB.  Antennas rotated so they are aligned N-S and cables N-S.  Similar to above but with noise in upper layers.}{2021-12-29 18:23:29.000}{6,6,6,6}{30,20,10,0}{2.000}{20000000.000}{40000000.000}{4}{4}{1}{127}{12.335}
\apresdoc{2021-12-29\_183509.dat}{Neumayer III,500m W of Station}{Additional Tx and Rx attenuator 10dB.  Antennas rotated so they are aligned N-S and cables N-S.  Similar to above but with noise in upper layers - repeat of abvoe checking connections and cables.  Still very low power.}{2021-12-29 18:35:38.000}{6,6,6,6}{30,20,10,0}{2.000}{20000000.000}{40000000.000}{4}{4}{1}{127}{12.391}
\apresdoc{2021-12-29\_184106.dat}{Neumayer III,500m W of Station}{Additional Rx attenuator 10dB.  Remove 10dB Tx attenuator.  Antennas rotated so they are aligned N-S and cables N-S.  Similar to above but with noise in upper layers - repeat of abvoe checking connections and cables.  Still very low power.}{2021-12-29 18:41:40.000}{6,6,6,6}{30,20,10,0}{2.000}{20000000.000}{40000000.000}{4}{4}{1}{127}{12.355}
\apresdoc{2021-12-29\_184802.dat}{Neumayer III,500m W of Station}{No additional attenuations. Antennas rotated so they are aligned N-S and cables N-S.  Similar to above but with noise in upper layers - repeat of abvoe checking connections and cables.  Overall increase in power - check consistency with removed attenuator.}{2021-12-29 18:48:39.000}{6,6,6,6}{30,20,10,0}{2.000}{20000000.000}{40000000.000}{4}{4}{1}{127}{12.347}
\apresdoc{2021-12-29\_185628.dat}{Neumayer III,500m W of Station}{Reset configuration to 1s.  No additional attenuations. Antennas rotated so they are aligned N-S and cables N-S.  Similar to above but with noise in upper layers.}{2021-12-29 18:57:01.000}{6,6,6,6}{30,20,10,0}{1.000}{20000000.000}{40000000.000}{4}{4}{1}{127}{12.310}
\apresdoc{2021-12-29\_190341.dat}{Neumayer III,500m W of Station}{No additional attenuations. Antennas rotated so they are aligned N-S and cables N-S.  Similar to above but with noise in upper layers. Move antennas closer to 10m separation but notable for RF 0dB.}{2021-12-29 19:03:47.000}{6,6,6,6}{30,20,10,0}{1.000}{20000000.000}{40000000.000}{4}{4}{1}{127}{12.303}
\apresdoc{2021-12-29\_190846.dat}{Neumayer III,500m W of Station}{No additional attenuations. Antennas rotated so they are aligned N-S and cables N-S.  Similar to above but with noise in upper layers. Reset antennas to 15m separation.  Otherwise as above.  Now clipping again (change in config?).}{2021-12-29 19:09:08.000}{6,6,6,6}{30,20,10,0}{1.000}{20000000.000}{40000000.000}{4}{4}{1}{127}{12.315}
\apresdoc{2021-12-29\_191455.dat}{Neumayer III,500m W of Station}{No additional attenuations. Antennas rotated so they are aligned N-S and cables N-S.  Similar to above but with noise in upper layers. Reset antennas to 15m separation.  Shake water from Tx and realign antennas to be more parallel.}{2021-12-29 19:15:00.000}{6,6,6,6}{30,20,10,0}{1.000}{20000000.000}{40000000.000}{4}{4}{1}{127}{12.319}
\apresdoc{2021-12-29\_191933.dat}{Neumayer III,500m W of Station}{No additional attenuations. Antennas rotated so they are aligned N-S and cables N-S.  Similar to above but with noise in upper layers. Reset antennas to 9m separation. }{2021-12-29 19:19:39.000}{6,6,6,6}{30,20,10,0}{1.000}{20000000.000}{40000000.000}{4}{4}{1}{127}{12.315}
\apresdoc{2021-12-29\_192217.dat}{Neumayer III,500m W of Station}{No additional attenuations. Antennas rotated so they are aligned N-S and cables N-S.  Similar to above but with noise in upper layers. 9m separation.  Cable moved from short bent loop to larger, smoother curve.  Reduced clipping?}{2021-12-29 19:22:34.000}{6,6,6,6}{30,20,10,0}{1.000}{20000000.000}{40000000.000}{4}{4}{1}{127}{12.315}
\apresdoc{2021-12-29\_193020.dat}{Neumayer III,500m W of Station}{No additional attenuations. Antennas rotated so they are aligned N-S and cables N-S.  Similar to above but with noise in upper layers. 9m separation.  Cable moved returned to sharp loop at end of each antenna.}{2021-12-29 19:30:33.000}{6,6,6,6}{30,20,10,0}{1.000}{20000000.000}{40000000.000}{4}{4}{1}{127}{12.315}
\apresdoc{2021-12-29\_193125.dat}{Neumayer III,500m W of Station}{No additional attenuations. Antennas rotated so they are aligned N-S and cables N-S.  Similar to above but with noise in upper layers. 9m separation.  Cable moved to smooth loop.}{2021-12-29 19:31:29.000}{6,6,6,6}{30,20,10,0}{1.000}{20000000.000}{40000000.000}{4}{4}{1}{127}{12.306}
\apresdoc{2021-12-29\_193232.dat}{Neumayer III,500m W of Station}{No additional attenuations. Antennas rotated so they are aligned N-S and cables N-S.  Similar to above but with noise in upper layers. 9m separation.  Cable moved to sharp loop.}{2021-12-29 19:32:50.000}{6,6,6,6}{30,20,10,0}{1.000}{20000000.000}{40000000.000}{4}{4}{1}{127}{12.315}
\apresdoc{2021-12-29\_193352.dat}{Neumayer III,500m W of Station}{No additional attenuations. Antennas rotated so they are aligned N-S and cables N-S.  Similar to above but with noise in upper layers. 9m separation.  Cable moved to smooth loop.}{2021-12-29 19:33:59.000}{6,6,6,6}{30,20,10,0}{1.000}{20000000.000}{40000000.000}{4}{4}{1}{127}{12.306}
\apresdoc{2021-12-29\_193505.dat}{Neumayer III,500m W of Station}{No additional attenuations. Antennas rotated so they are aligned N-S and cables N-S.  Similar to above but with noise in upper layers. 9m separation.  Cable moved to sharp loop.}{2021-12-29 19:35:11.000}{6,6,6,6}{30,20,10,0}{1.000}{20000000.000}{40000000.000}{4}{4}{1}{127}{12.310}
\apresdoc{2021-12-29\_214056.dat}{Neumayer III,500m W of Station}{No additional attenuations. Antennas rotated so they are aligned N-S and cables N-S.  Similar to above but with noise in upper layers. 9m separation.  Return to setup after dinner.  Otherwise unchanged. }{2021-12-29 21:42:04.000}{6,6,6,6}{30,20,10,0}{1.000}{20000000.000}{40000000.000}{4}{4}{1}{127}{12.323}
\apresdoc{2021-12-29\_214443.dat}{Neumayer III,500m W of Station}{No additional attenuations. Antennas rotated so they are aligned N-S and cables N-S.  Similar to above but with noise in upper layers. 9m separation.  Return to setup after dinner.  Reduced power code to 0.}{2021-12-29 21:44:51.000}{6,6,6,6}{30,20,10,0}{1.000}{20000000.000}{40000000.000}{4}{4}{1}{0}{12.282}
\apresdoc{2021-12-29\_215406.dat}{Neumayer III,500m W of Station}{No additional attenuations. Antennas rotated so they are aligned N-S and cables N-S.  Similar to above but with noise in upper layers. 9m separation.  Cables 'smooth'.  Reduced power code to 0.}{2021-12-29 21:54:27.000}{6,6,6,6}{30,20,10,0}{1.000}{20000000.000}{40000000.000}{4}{4}{1}{0}{12.270}
\apresdoc{2021-12-29\_215659.dat}{Neumayer III,500m W of Station}{No additional attenuations. Antennas rotated so they are aligned N-S and cables N-S.  Similar to above but with noise in upper layers. 14m separation.  Cables 'smooth' in line with antennas.  Reduced power code to 0.}{2021-12-29 21:58:31.000}{6,6,6,6}{30,20,10,0}{1.000}{20000000.000}{40000000.000}{4}{4}{1}{0}{12.266}
\apresdoc{2021-12-29\_220442.dat}{Neumayer III,500m W of Station}{No additional attenuations. Antennas rotated so they are aligned E-W but Tx at N and Rx at S.  Similar to above but with noise in upper layers. 14m separation.  Cables 'smooth' in line with antennas.  Reduced power code to 0.}{2021-12-29 22:04:45.000}{6,6,6,6}{30,20,10,0}{1.000}{20000000.000}{40000000.000}{4}{4}{1}{0}{12.258}
\apresdoc{2021-12-29\_220927.dat}{Neumayer III,500m W of Station}{No additional attenuations. Antennas rotated so they are aligned N-S along N-S profile.  Similar to above but with noise in upper layers. 14m separation.  Cables 'smooth' in line with antennas.  Reduced power code to 0.}{2021-12-29 22:09:34.000}{6,6,6,6}{30,20,10,0}{1.000}{20000000.000}{40000000.000}{4}{4}{1}{0}{12.254}
\apresdoc{1\_SubZero\_\_211150.30\_T1HR1H.dat}{Groundling Line Camp,-71.21423633N,-08.5932860E}{No comments noted.}{2021-12-30 22:13:11.000}{-14}{30}{1.000}{20000000.000}{40000000.000}{1}{20}{5}{127}{12.379}
\apresdoc{1\_SubZero\_\_164126.90\_T1HR1H.dat}{Groundling Line Camp,-71.21423633N,-08.5932860E}{50m from mcamp.  No basal reflector - expect to see at around 1050m.}{2022-01-02 17:42:48.000}{-14}{30}{1.000}{20000000.000}{40000000.000}{1}{20}{5}{127}{12.222}
\apresdoc{1\_SubZero\_\_165816.20\_T1HR1H.dat}{Groundling Line Camp,-71.21423633N,-08.5932860E}{50m from mcamp.  No basal reflector - expect to see at around 1050m.}{2022-01-02 17:59:38.000}{-14}{30}{1.000}{20000000.000}{40000000.000}{1}{20}{5}{127}{12.363}
\apresdoc{1\_SubZero\_\_180215.40\_T1HR1H.dat}{Groundling Line Camp,-71.21423633N,-08.5932860E}{50m from mcamp.  No basal reflector - expect to see at around 1050m.}{2022-01-02 18:02:20.000}{6}{30}{0.100}{20000000.000}{40000000.000}{1}{5}{5}{127}{12.274}
\apresdoc{2\_SubZero\_\_180505.20\_T1HR1H.dat}{Groundling Line Camp,-71.21423633N,-08.5932860E}{50m from camp.  No basal reflector - expect to see at around 1050m.}{2022-01-02 18:05:09.000}{6}{30}{0.100}{20000000.000}{40000000.000}{1}{300}{300}{127}{12.274}
\apresdoc{1\_SubZero\_\_181539.50\_T1HR1H.dat}{Groundling Line Camp,-71.21423633N,-08.5932860E}{50m from camp.  No basal reflector - expect to see at around 1050m.  Large amount of transient noise - low power.}{2022-01-02 18:15:40.000}{-14}{30}{1.000}{20000000.000}{40000000.000}{1}{300}{300}{127}{12.262}
\apresdoc{2022-01-02\_182206.dat}{Groundling Line Camp,-71.21423633N,-08.5932860E}{50m from camp.  No basal reflector - expect to see at around 1050m.  Clipping over lower bandwidth.}{2022-01-02 18:23:08.000}{6,6,6,6}{30,20,10,0}{1.000}{20000000.000}{40000000.000}{4}{20}{5}{127}{12.234}
\apresdoc{2022-01-02\_182915.dat}{Groundling Line Camp,-71.21423633N,-08.5932860E}{50m from camp.  No basal reflector - expect to see at around 1050m.  Reduced period - clipping now occupies entire signal - transient effect?}{2022-01-02 18:30:04.000}{6,6,6,6}{30,20,10,0}{0.100}{20000000.000}{40000000.000}{4}{40}{10}{127}{12.254}
\apresdoc{2022-01-02\_183717.dat}{Groundling Line Camp,-71.21423633N,-08.5932860E}{Additional 10dB Rx attenuator. 50m from camp.  No basal reflector - expect to see at around 1050m.  Reduced period - clipping now occupies entire signal - transient effect?}{2022-01-02 18:38:08.000}{6,6,6,6}{30,20,10,0}{0.100}{20000000.000}{40000000.000}{4}{40}{10}{127}{12.266}
\apresdoc{2022-01-02\_185552.dat}{Groundling Line Camp}{Additional 10dB Rx attenuator. 50m from camp.  No basal reflector - expect to see at around 1050m.  Increased period back to 1s.  Antennas pulled and location changed.}{2022-01-02 18:57:56.000}{6,6,6,6}{30,20,10,0}{1.000}{20000000.000}{40000000.000}{4}{4}{1}{127}{12.250}
\apresdoc{2022-01-02\_185854.dat}{Groundling Line Camp}{Additional 10dB Rx attenuator. 50m from camp.  Receive antenna removed.}{2022-01-02 18:59:14.000}{6,6,6,6}{30,20,10,0}{1.000}{20000000.000}{40000000.000}{4}{4}{1}{127}{12.246}
\apresdoc{2022-01-02\_191302.dat}{Groundling Line Camp}{Additional 10dB Rx attenuator. 50m from camp.  Receive antenna removed and transmit antenna terminated.}{2022-01-02 19:14:08.000}{6,6,6,6}{30,20,10,0}{1.000}{20000000.000}{40000000.000}{4}{4}{1}{127}{12.258}
\apresdoc{2022-01-03\_115005.dat}{Groundling Line Camp}{Additional 20dB attenuator on Tx.  No sign of base.}{2022-01-03 11:50:45.000}{6,6,6,6}{30,20,10,0}{1.000}{20000000.000}{40000000.000}{4}{80}{20}{127}{12.444}
\apresdoc{2022-01-03\_144007.dat}{Groundling Line Camp}{Additional 20dB attenuator on Tx.  No sign of base.  Replace F003 board with F002.}{2022-01-03 14:43:43.000}{6,6,6,6}{30,20,10,0}{1.000}{20000000.000}{40000000.000}{4}{4}{1}{127}{12.375}
\apresdoc{2022-01-03\_152109.dat}{Groundling Line Camp}{Additional 20dB attenuator on Tx.  No sign of base.  Replace F003 board with F002.  Move antennas from 10m to 20m separation.}{2022-01-03 15:22:23.000}{6,6,6,6}{30,20,10,0}{1.000}{20000000.000}{40000000.000}{4}{40}{10}{127}{12.306}
\apresdoc{2022-01-03\_153752.dat}{Groundling Line Camp}{Additional 20dB attenuator on Tx.  No sign of base.  Board F002.  Move antennas from 10m to 20m separation. Reduced power code to 0.}{2022-01-03 15:39:10.000}{6,6,6,6}{30,20,10,0}{1.000}{20000000.000}{40000000.000}{4}{40}{10}{0}{12.282}
\apresdoc{2022-01-03\_161929.dat}{Groundling Line Camp}{Additional 20dB attenuator on Tx.  No sign of base.  Board F002.  Move antennas from 10m to 20m separation. Reduced power code to 0.  Move Hilux further away.}{2022-01-03 16:19:56.000}{6,6,6,6}{30,20,10,0}{1.000}{20000000.000}{40000000.000}{4}{40}{10}{127}{12.395}
\apresdoc{2022-01-03\_164331.dat}{Groundling Line Camp,-71.72445466,-08.59249550,135.2m Elevation}{Additional 20dB attenuator on Tx.  No sign of base.  Board F002.  Move antennas from 10m to 20m separation. Try short survey to find base.}{2022-01-03 16:43:55.000}{-14}{10}{1.000}{20000000.000}{40000000.000}{1}{1}{1}{127}{12.226}
\apresdoc{2022-01-03\_170023.dat}{Groundling Line Camp,-71.72445466,-08.59249550,135.2m Elevation}{Additional 20dB attenuator on Tx.  See basal reflector with >4 summed chirps.}{2022-01-03 17:00:45.000}{-14}{30}{1.000}{20000000.000}{40000000.000}{1}{40}{40}{127}{12.218}
\apresdoc{2022-01-03\_173640.dat}{Groundling Line Camp,-71.72444549,-08.59246611,132.2m Elevation}{Additional 20dB attenuator on Tx.  Basal reflector disappeared.  Moved 1m from previous site.}{2022-01-03 17:36:56.000}{-14}{30}{1.000}{20000000.000}{40000000.000}{1}{20}{20}{127}{12.186}
\apresdoc{2022-01-03\_174337.dat}{Groundling Line Camp,-71.72443744,-08.59244250,132m Elevation}{Additional 20dB attenuator on Tx.  Basal reflector disappeared.  Moved 1m from previous site.}{2022-01-03 17:43:56.000}{-14}{30}{1.000}{20000000.000}{40000000.000}{1}{20}{20}{127}{12.182}
\apresdoc{2022-01-03\_174745.dat}{Groundling Line Camp,-71.72443744,-08.59244250,132m Elevation}{Additional 20dB attenuator on Tx.  Basal reflector disappeared.  As previous site with reduced power code.}{2022-01-03 17:48:19.000}{-14}{30}{1.000}{20000000.000}{40000000.000}{1}{20}{20}{0}{12.178}
\apresdoc{2022-01-03\_175416.dat}{Groundling Line Camp,-71.7244400,-08.59241450,132m Elevation}{Additional 20dB attenuator on Tx.  Basal reflector disappeared.  Moved 1m from previous position.}{2022-01-03 17:56:13.000}{-14}{30}{1.000}{20000000.000}{40000000.000}{1}{40}{40}{0}{12.170}
\apresdoc{2022-01-03\_180431.dat}{Groundling Line Camp,-71.7244400,-08.59241450,132m Elevation}{Additional 20dB attenuator on Tx.  Basal reflector disappeared.  In same position - release cables.}{2022-01-03 18:04:41.000}{-14}{30}{1.000}{20000000.000}{40000000.000}{1}{40}{40}{0}{12.165}
\apresdoc{2022-01-03\_192615.dat}{Groundling Line Camp,-71.7244400,-08.59241450,132m Elevation}{Additional 20dB attenuator on Tx.  Basal reflector disappeared.  Left ApRES for an hour an a half.  Ground cooled and cables frozen in.}{2022-01-03 19:28:49.000}{-14}{30}{1.000}{20000000.000}{40000000.000}{1}{40}{40}{0}{12.162}
\apresdoc{2022-01-03\_194439.dat}{Groundling Line Camp,-71.7244400,-08.59241450,132m Elevation}{Additional 20dB attenuator on Tx.  Basal reflector possible to see with some averaging?}{2022-01-03 19:45:03.000}{-4}{30}{1.000}{20000000.000}{40000000.000}{1}{60}{60}{127}{12.129}
\apresdoc{DATA2022-01-03-1957.DAT}{Groundling Line Camp,-71.7244400,-08.59241450,132m Elevation}{Averaging every 5 minutes every 1 hours.}{2022-01-03 19:57:41.000}{-4}{30}{1.000}{20000000.000}{40000000.000}{1}{60}{60}{127}{12.198}
\apresdoc{DATA2022-01-03-2002.DAT}{Groundling Line Camp,-71.7244400,-08.59241450,132m Elevation}{Averaging every 5 minutes every 1 hours.}{2022-01-03 20:02:26.000}{-4}{30}{1.000}{20000000.000}{40000000.000}{1}{60}{60}{127}{12.310}
\apresdoc{DATA2022-01-03-2007.DAT}{Groundling Line Camp,-71.7244400,-08.59241450,132m Elevation}{Averaging every 5 minutes every 1 hours.}{2022-01-03 20:07:26.000}{-4}{30}{1.000}{20000000.000}{40000000.000}{1}{60}{60}{127}{12.315}
\apresdoc{DATA2022-01-03-2012.DAT}{Groundling Line Camp,-71.7244400,-08.59241450,132m Elevation}{Averaging every 5 minutes every 1 hours.}{2022-01-03 20:12:26.000}{-4}{30}{1.000}{20000000.000}{40000000.000}{1}{60}{60}{127}{12.310}
\apresdoc{DATA2022-01-03-2017.DAT}{Groundling Line Camp,-71.7244400,-08.59241450,132m Elevation}{Averaging every 5 minutes every 1 hours.}{2022-01-03 20:17:26.000}{-4}{30}{1.000}{20000000.000}{40000000.000}{1}{60}{60}{127}{12.306}
\apresdoc{DATA2022-01-03-2022.DAT}{Groundling Line Camp,-71.7244400,-08.59241450,132m Elevation}{Averaging every 5 minutes every 1 hours.}{2022-01-03 20:22:26.000}{-4}{30}{1.000}{20000000.000}{40000000.000}{1}{60}{60}{127}{12.306}
\apresdoc{DATA2022-01-03-2027.DAT}{Groundling Line Camp,-71.7244400,-08.59241450,132m Elevation}{Averaging every 5 minutes every 1 hours.}{2022-01-03 20:27:26.000}{-4}{30}{1.000}{20000000.000}{40000000.000}{1}{60}{60}{127}{12.303}
\apresdoc{DATA2022-01-03-2032.DAT}{Groundling Line Camp,-71.7244400,-08.59241450,132m Elevation}{Averaging every 5 minutes every 1 hours.}{2022-01-03 20:32:26.000}{-4}{30}{1.000}{20000000.000}{40000000.000}{1}{60}{60}{127}{12.299}
\apresdoc{DATA2022-01-03-2037.DAT}{Groundling Line Camp,-71.7244400,-08.59241450,132m Elevation}{Averaging every 5 minutes every 1 hours.}{2022-01-03 20:37:26.000}{-4}{30}{1.000}{20000000.000}{40000000.000}{1}{60}{60}{127}{12.299}
\apresdoc{DATA2022-01-03-2042.DAT}{Groundling Line Camp,-71.7244400,-08.59241450,132m Elevation}{Averaging every 5 minutes every 1 hours.}{2022-01-03 20:42:26.000}{-4}{30}{1.000}{20000000.000}{40000000.000}{1}{60}{60}{127}{12.294}
\apresdoc{DATA2022-01-03-2047.DAT}{Groundling Line Camp,-71.7244400,-08.59241450,132m Elevation}{Averaging every 5 minutes every 1 hours.}{2022-01-03 20:47:26.000}{-4}{30}{1.000}{20000000.000}{40000000.000}{1}{60}{60}{127}{12.290}
\apresdoc{DATA2022-01-03-2052.DAT}{Groundling Line Camp,-71.7244400,-08.59241450,132m Elevation}{Averaging every 5 minutes every 1 hours.}{2022-01-03 20:52:26.000}{-4}{30}{1.000}{20000000.000}{40000000.000}{1}{60}{60}{127}{12.282}
\apresdoc{DATA2022-01-03-2057.DAT}{Groundling Line Camp,-71.7244400,-08.59241450,132m Elevation}{Averaging every 5 minutes every 1 hours.}{2022-01-03 20:57:26.000}{-4}{30}{1.000}{20000000.000}{40000000.000}{1}{60}{60}{127}{12.282}
\apresdoc{DATA2022-01-03-2102.DAT}{Groundling Line Camp,-71.7244400,-08.59241450,132m Elevation}{Averaging every 5 minutes every 1 hours.}{2022-01-03 21:02:26.000}{-4}{30}{1.000}{20000000.000}{40000000.000}{1}{60}{60}{127}{12.278}
\apresdoc{DATA2022-01-03-2107.DAT}{Groundling Line Camp,-71.7244400,-08.59241450,132m Elevation}{Averaging every 5 minutes every 1 hours.}{2022-01-03 21:07:26.000}{-4}{30}{1.000}{20000000.000}{40000000.000}{1}{60}{60}{127}{12.278}
\apresdoc{DATA2022-01-03-2112.DAT}{Groundling Line Camp,-71.7244400,-08.59241450,132m Elevation}{Averaging every 5 minutes every 1 hours.}{2022-01-03 21:12:26.000}{-4}{30}{1.000}{20000000.000}{40000000.000}{1}{60}{60}{127}{12.278}
\apresdoc{DATA2022-01-03-2117.DAT}{Groundling Line Camp,-71.7244400,-08.59241450,132m Elevation}{Averaging every 5 minutes every 1 hours.}{2022-01-03 21:17:26.000}{-4}{30}{1.000}{20000000.000}{40000000.000}{1}{60}{60}{127}{12.270}
\apresdoc{DATA2022-01-03-2122.DAT}{Groundling Line Camp,-71.7244400,-08.59241450,132m Elevation}{Averaging every 5 minutes every 1 hours.}{2022-01-03 21:22:26.000}{-4}{30}{1.000}{20000000.000}{40000000.000}{1}{60}{60}{127}{12.266}
\apresdoc{DATA2022-01-04-0805.DAT}{Groundling Line Camp,-71.7244400,-08.59241450,132m Elevation}{Averaging every 5 minutes every 1 hours.}{2022-01-04 08:05:35.000}{-4}{30}{1.000}{20000000.000}{40000000.000}{1}{60}{60}{127}{12.854}
\apresdoc{DATA2022-01-04-0810.DAT}{Groundling Line Camp,-71.7244400,-08.59241450,132m Elevation}{Averaging every 5 minutes every 1 hours.}{2022-01-04 08:10:20.000}{-4}{30}{1.000}{20000000.000}{40000000.000}{1}{60}{60}{127}{12.786}
\apresdoc{DATA2022-01-04-0815.DAT}{Groundling Line Camp,-71.7244400,-08.59241450,132m Elevation}{Averaging every 5 minutes every 1 hours.}{2022-01-04 08:15:20.000}{-4}{30}{1.000}{20000000.000}{40000000.000}{1}{60}{60}{127}{12.758}
\apresdoc{DATA2022-01-04-0820.DAT}{Groundling Line Camp,-71.7244400,-08.59241450,132m Elevation}{Averaging every 5 minutes every 1 hours.}{2022-01-04 08:20:20.000}{-4}{30}{1.000}{20000000.000}{40000000.000}{1}{60}{60}{127}{12.729}
\apresdoc{DATA2022-01-04-0825.DAT}{Groundling Line Camp,-71.7244400,-08.59241450,132m Elevation}{Averaging every 5 minutes every 1 hours.}{2022-01-04 08:25:20.000}{-4}{30}{1.000}{20000000.000}{40000000.000}{1}{60}{60}{127}{12.705}
\apresdoc{DATA2022-01-04-0830.DAT}{Groundling Line Camp,-71.7244400,-08.59241450,132m Elevation}{Averaging every 5 minutes every 1 hours.}{2022-01-04 08:30:20.000}{-4}{30}{1.000}{20000000.000}{40000000.000}{1}{60}{60}{127}{12.681}
\apresdoc{DATA2022-01-04-0835.DAT}{Groundling Line Camp,-71.7244400,-08.59241450,132m Elevation}{Averaging every 5 minutes every 1 hours.}{2022-01-04 08:35:20.000}{-4}{30}{1.000}{20000000.000}{40000000.000}{1}{60}{60}{127}{12.665}
\apresdoc{DATA2022-01-04-0840.DAT}{Groundling Line Camp,-71.7244400,-08.59241450,132m Elevation}{Averaging every 5 minutes every 1 hours.}{2022-01-04 08:40:20.000}{-4}{30}{1.000}{20000000.000}{40000000.000}{1}{60}{60}{127}{12.649}
\apresdoc{DATA2022-01-04-0845.DAT}{Groundling Line Camp,-71.7244400,-08.59241450,132m Elevation}{Averaging every 5 minutes every 1 hours.}{2022-01-04 08:45:20.000}{-4}{30}{1.000}{20000000.000}{40000000.000}{1}{60}{60}{127}{12.633}
\apresdoc{DATA2022-01-04-0850.DAT}{Groundling Line Camp,-71.7244400,-08.59241450,132m Elevation}{Averaging every 5 minutes every 1 hours.}{2022-01-04 08:50:20.000}{-4}{30}{1.000}{20000000.000}{40000000.000}{1}{60}{60}{127}{12.621}
\apresdoc{DATA2022-01-04-0855.DAT}{Groundling Line Camp,-71.7244400,-08.59241450,132m Elevation}{Averaging every 5 minutes every 1 hours.}{2022-01-04 08:55:20.000}{-4}{30}{1.000}{20000000.000}{40000000.000}{1}{60}{60}{127}{12.609}
\apresdoc{DATA2022-01-04-0900.DAT}{Groundling Line Camp,-71.7244400,-08.59241450,132m Elevation}{Averaging every 5 minutes every 1 hours.}{2022-01-04 09:00:20.000}{-4}{30}{1.000}{20000000.000}{40000000.000}{1}{60}{60}{127}{12.605}
\apresdoc{DATA2022-01-04-0905.DAT}{Groundling Line Camp,-71.7244400,-08.59241450,132m Elevation}{Averaging every 5 minutes every 1 hours.}{2022-01-04 09:05:20.000}{-4}{30}{1.000}{20000000.000}{40000000.000}{1}{60}{60}{127}{12.605}
\apresdoc{DATA2022-01-04-0910.DAT}{Groundling Line Camp,-71.7244400,-08.59241450,132m Elevation}{Averaging every 5 minutes every 1 hours.}{2022-01-04 09:10:20.000}{-4}{30}{1.000}{20000000.000}{40000000.000}{1}{60}{60}{127}{12.597}
\apresdoc{DATA2022-01-04-0915.DAT}{Groundling Line Camp,-71.7244400,-08.59241450,132m Elevation}{Averaging every 5 minutes every 1 hours.}{2022-01-04 09:15:20.000}{-4}{30}{1.000}{20000000.000}{40000000.000}{1}{60}{60}{127}{12.597}
\apresdoc{DATA2022-01-04-0920.DAT}{Groundling Line Camp,-71.7244400,-08.59241450,132m Elevation}{Averaging every 5 minutes every 1 hours.}{2022-01-04 09:20:20.000}{-4}{30}{1.000}{20000000.000}{40000000.000}{1}{60}{60}{127}{12.592}
\apresdoc{2022-01-04\_110118.dat}{Groundling Line Camp}{No comments noted.}{2022-01-04 11:01:32.000}{-14}{30}{1.000}{20000000.000}{40000000.000}{1}{40}{40}{127}{12.427}
\apresdoc{2022-01-04\_133425.dat}{Groundling Line Camp}{20dB Tx attenuation.  Sled moved by 15m from position.  Mount antennas on tractor tyres.}{2022-01-04 13:34:59.000}{-14}{30}{1.000}{20000000.000}{40000000.000}{1}{40}{40}{127}{12.371}
\apresdoc{2022-01-04\_133729.dat}{Groundling Line Camp}{20dB Tx attenuation.  Sled moved by 15m from position.  Mount antennas on tractor tyres.}{2022-01-04 13:37:59.000}{-14,-4,6}{30,20,10}{1.000}{20000000.000}{40000000.000}{3}{120}{40}{127}{12.355}
\apresdoc{2022-01-04\_142639.dat}{Groundling Line Camp}{20dB Tx attenuation.  Sled moved by 30m from last position.  Mount antennas on tractor tyres.  }{2022-01-04 14:27:09.000}{-14,-4,6}{30,30,30}{1.000}{20000000.000}{40000000.000}{3}{3}{1}{127}{12.246}
\apresdoc{2022-01-04\_142723.dat}{Groundling Line Camp}{20dB Tx attenuation.  Sled moved back to original position of the morning.  Tyres removed.  Power code reset to 127.}{2022-01-04 14:27:35.000}{-14,-4,6}{30,30,30}{1.000}{20000000.000}{40000000.000}{3}{120}{40}{127}{12.218}
\apresdoc{2022-01-04\_153231.dat}{Groundling Line Camp}{20dB Tx attenuation.  Sled moved back to original position of the morning.  Tyres removed.  Power code reset to 127.}{2022-01-04 15:33:31.000}{-14,-14,-4,-4}{30,20,30,20}{1.000}{20000000.000}{40000000.000}{4}{160}{40}{127}{12.198}
\apresdoc{2022-01-05\_113341.dat}{Grounding Line Camp}{Additional attenuator 10dB Rx, 10dB Tx.  Antenna separation 20m still (phase-centres)?}{2022-01-05 11:34:36.000}{-4,-14}{30,30}{1.000}{20000000.000}{40000000.000}{2}{80}{40}{127}{12.468}
\apresdoc{2022-01-05\_121151.dat}{Grounding Line Camp}{Additional attenuator 10dB Rx, 10dB Tx.  Antenna separation 20m still (phase-centres)?  Reduced period shows clipping again is a transient effect?}{2022-01-05 12:12:33.000}{-4,-14}{30,30}{0.100}{20000000.000}{40000000.000}{2}{80}{40}{127}{12.460}
\apresdoc{2022-01-05\_122713.dat}{Grounding Line Camp}{Additional attenuator 10dB Rx, 10dB Tx.  Increase Tx cable to 25m and separate antennas to maximum length (35m?).  Reset period tot 1s.}{2022-01-05 12:28:07.000}{-14,-4}{30,30}{1.000}{20000000.000}{40000000.000}{2}{80}{40}{127}{12.427}
\apresdoc{2022-01-05\_123749.dat}{Grounding Line Camp}{Additional attenuator 10dB Rx, 10dB Tx.  Increase Tx cable to 25m and separate antennas to maximum length (35m?).  Repeat of previous.}{2022-01-05 12:38:03.000}{-14}{30}{1.000}{20000000.000}{40000000.000}{1}{40}{40}{127}{12.419}
\apresdoc{2022-01-05\_124357.dat}{Grounding Line Camp}{Additional attenuator 10dB Rx, 0dB Tx.  Increase Tx cable to 25m and separate antennas to maximum length (35m?).  Repeat of previous.  Possible bed signal?}{2022-01-05 12:46:01.000}{-14}{30}{1.000}{20000000.000}{40000000.000}{1}{40}{40}{127}{12.411}
\apresdoc{2022-01-05\_140022.dat}{Grounding Line Camp}{Additional attenuator 10dB Rx, 0dB Tx.  Increase Tx cable to 25m and separate antennas to maximum length (35m?).  Repeat of previous after lunch.  PulseEKKO skiddoo at end of line of profile (maybe 50m away - clipping in RF stage?)}{2022-01-05 14:00:42.000}{-14}{30}{1.000}{20000000.000}{40000000.000}{1}{40}{40}{127}{12.262}
\apresdoc{2022-01-05\_141217.dat}{Grounding Line Camp}{Additional attenuator 10dB Rx, 0dB Tx.  Increase Tx cable to 25m and separate antennas to maximum length (35m?).  Repeat of previous after lunch.  Skiddoo moved to within camp and repeated.}{2022-01-05 14:13:12.000}{-14}{30}{1.000}{20000000.000}{40000000.000}{1}{40}{40}{127}{12.359}
\apresdoc{2022-01-05\_142157.dat}{Grounding Line Camp}{Additional attenuator 10dB Rx, 0dB Tx.  Increase Tx cable to 25m and separate antennas to maximum length (35m?).  Try longer duration (5s), again base possible to see, maybe...}{2022-01-05 14:22:24.000}{-14}{30}{5.000}{20000000.000}{40000000.000}{1}{8}{8}{127}{12.347}
\apresdoc{2022-01-05\_142609.dat}{Grounding Line Camp}{Additional attenuator 10dB Rx, 0dB Tx.  Increase Tx cable to 25m and separate antennas to maximum length (35m?).  Should have used more subbursts.}{2022-01-05 14:26:43.000}{-14,-4,6}{30,30,30}{5.000}{20000000.000}{40000000.000}{3}{3}{1}{127}{12.339}
\apresdoc{2022-01-05\_142942.dat}{Grounding Line Camp}{Additional attenuator 10dB Rx, 0dB Tx.  Increase Tx cable to 25m and separate antennas to maximum length (35m?).  High frequency noise evident in signal - radio operation? Later note that skiddo is nearby? Could be PulseEKKO?}{2022-01-05 14:30:53.000}{-14,-14,-14,-14}{30,20,10,0}{5.000}{20000000.000}{40000000.000}{4}{4}{1}{127}{12.335}
\apresdoc{2022-01-05\_145817.dat}{Grounding Line Camp}{Additional attenuator 10dB Rx, 0dB Tx.  Increase Tx cable to 25m (Rx 5m) and separate antennas to maximum length (30m).  Reset config to 1s.  High frequency noise disappears. Skiddoo moving nearby and cable in middle of antennas.}{2022-01-05 14:58:45.000}{-14}{30}{1.000}{20000000.000}{40000000.000}{1}{40}{40}{127}{12.282}
\apresdoc{2022-01-05\_150400.dat}{Grounding Line Camp}{Additional attenuator 10dB Rx, 0dB Tx.  Increase Tx cable to 25m (Rx 5m) and separate antennas to maximum length (30m).  Reset config to 1s.  High frequency noise disappears. Skiddoo back in camp and cable moved.}{2022-01-05 15:04:25.000}{-14}{30}{1.000}{20000000.000}{40000000.000}{1}{40}{40}{127}{12.274}
\apresdoc{2022-01-05\_152208.dat}{Grounding Line Camp}{Additional attenuator 10dB Rx, 0dB Tx.  Increase Tx cable to 25m (Rx 5m) and separate antennas to maximum length (30m).  Reset config to 1s.  Assembly moved further from camp.  Rx  connection loose.}{2022-01-05 15:23:23.000}{-14}{30}{1.000}{20000000.000}{40000000.000}{1}{40}{40}{127}{12.234}
\apresdoc{2022-01-05\_153218.dat}{Grounding Line Camp}{Additional attenuator 10dB Rx, 0dB Tx.  Increase Tx cable to 25m (Rx 5m) and separate antennas to maximum length (30m).  Reset config to 1s.  Assembly moved further from camp.  Rx connection replaced.}{2022-01-05 15:33:05.000}{-14}{30}{1.000}{20000000.000}{40000000.000}{1}{40}{40}{127}{12.218}
\apresdoc{2022-01-05\_155959.dat}{Grounding Line Camp}{Additional attenuator 10dB Rx, 0dB Tx.  Increase Tx cable to 25m (Rx 5m) and separate antennas to maximum length (30m).  Checking cabling - battery voltage fallen to 12.1?}{2022-01-05 16:00:36.000}{-14}{30}{1.000}{20000000.000}{40000000.000}{1}{40}{40}{127}{12.178}
\apresdoc{2022-01-05\_163611.dat}{Grounding Line Camp}{Additional attenuator 10dB Rx, 0dB Tx.  Increase Tx cable to 25m (Rx 5m) and separate antennas to maximum length (30m).  Maybe bed possible to see here?}{2022-01-05 16:36:52.000}{-14}{30}{1.000}{20000000.000}{40000000.000}{1}{40}{40}{127}{12.286}
\apresdoc{2022-01-05\_164353.dat}{Grounding Line Camp}{Additional attenuator 10dB Rx, 0dB Tx.  Increase Tx cable to 25m (Rx 5m) and separate antennas to maximum length (30m).  Antennas rotated broadside.  Bed clearer (higher SNR)? }{2022-01-05 16:44:58.000}{-14}{0}{1.000}{20000000.000}{40000000.000}{1}{40}{40}{127}{12.258}
\apresdoc{2022-01-05\_164908.dat}{Grounding Line Camp}{Additional attenuator 10dB Rx, 0dB Tx.  Increase Tx cable to 25m (Rx 5m) and separate antennas to maximum length (30m).  Antennas rotated broadside.  Clipping in Tx with increased AF gain.}{2022-01-05 16:49:21.000}{6}{0}{1.000}{20000000.000}{40000000.000}{1}{40}{40}{127}{12.254}
\apresdoc{2022-01-05\_165236.dat}{Grounding Line Camp}{Additional attenuator 10dB Rx, 0dB Tx.  Increase Tx cable to 25m (Rx 5m) and separate antennas to maximum length (30m).  Antennas rotated broadside.  Bed clearer (higher SNR)? Clipping in Tx with increased AF gain - at earlier stages in RF chain?}{2022-01-05 16:53:50.000}{-4}{0}{1.000}{20000000.000}{40000000.000}{1}{40}{40}{127}{12.246}
\apresdoc{2022-01-05\_165821.dat}{Grounding Line Camp}{Additional attenuator 10dB Rx, 0dB Tx.  Increase Tx cable to 25m (Rx 5m) and separate antennas to maximum length (30m).  Antennas rotated broadside.  Bed clearer (higher SNR)? Some clipping in signal.  Cabled moved from last measurement - ringing?}{2022-01-05 16:59:32.000}{-14,-4,6}{0,0,0}{1.000}{20000000.000}{40000000.000}{3}{120}{40}{127}{12.246}
\apresdoc{2022-01-05\_171029.dat}{Grounding Line Camp}{Additional attenuator 10dB Rx, 0dB Tx.  Increase Tx cable to 25m (Rx 5m) and separate antennas to maximum length (30m).  Antennas rotated broadside.  Bed clearer (higher SNR)? Some clipping in signal.  Repeated as before - ringing reduced?}{2022-01-05 17:10:51.000}{-14,-4,6}{0,0,0}{1.000}{20000000.000}{40000000.000}{3}{120}{40}{127}{12.238}
\apresdoc{2022-01-05\_174143.dat}{Grounding Line Camp}{Additional attenuator 10dB Rx, 0dB Tx.  Cable length Tx 25m, Rx 25m and separate antennas to maximum length (30m).  Antennas return to endfire.  Clipping across all AF settings.}{2022-01-05 17:42:41.000}{-14,-4,6}{0,0,0}{1.000}{20000000.000}{40000000.000}{3}{3}{1}{127}{12.254}
\apresdoc{2022-01-05\_174254.dat}{Grounding Line Camp}{Additional attenuator 10dB Rx, 0dB Tx.  Cable length Tx 25m, Rx 25m and separate antennas to maximum length (30m).  Antennas return to endfire.  Clipping across all AF settings.  No clear basal return.}{2022-01-05 17:43:31.000}{-14,-4,6}{0,0,0}{1.000}{20000000.000}{40000000.000}{3}{120}{40}{127}{12.238}
\apresdoc{2022-01-05\_175045.dat}{Grounding Line Camp}{Additional attenuator 10dB Rx, 0dB Tx.  Cable length Tx 25m, Rx 25m and separate antennas to maximum length (30m).  Antennas return to endfire.  Clipping across all AF settings.  No clear basal return.}{2022-01-05 17:55:13.000}{-14,-4,6}{0,0,0}{1.000}{20000000.000}{40000000.000}{3}{120}{40}{127}{12.230}
\apresdoc{2022-01-05\_184909.dat}{Grounding Line Camp}{Additional attenuator 10dB Rx, 0dB Tx.  Cable length Tx 25m, Rx 25m and separate antennas to maximum length (40m).  Antennas broadside.  Reduced clipping through signal.}{2022-01-05 18:51:05.000}{-14,-4,6}{0,0,0}{1.000}{20000000.000}{40000000.000}{3}{120}{40}{127}{12.218}
\apresdoc{2022-01-05\_190323.dat}{Grounding Line Camp}{Additional attenuator 10dB Rx, 0dB Tx.  Cable length Tx 25m, Rx 25m and separate antennas to maximum length (40m).  Antennas broadside.  Repeat as previous - base visible again?}{2022-01-05 19:03:59.000}{-14,-4,6}{0,0,0}{1.000}{20000000.000}{40000000.000}{3}{120}{40}{127}{12.218}
\apresdoc{2022-01-05\_234755.dat}{Grounding Line Camp}{Additional attenuator 10dB Rx, 0dB Tx.  Cable length Tx 25m, Rx 25m and separate antennas to maximum length (40m).  Antennas broadside.  Repeat as previous - base visible again? Generator still on?}{2022-01-05 23:48:28.000}{-14,-4,6}{0,0,0}{1.000}{20000000.000}{40000000.000}{3}{120}{40}{127}{12.669}
\apresdoc{2022-01-05\_235346.dat}{Grounding Line Camp}{Additional attenuator 10dB Rx, 0dB Tx.  Cable length Tx 25m, Rx 25m and separate antennas to maximum length (40m).  Antennas broadside.  Repeat as previous - base visible again? Generator turned off.}{2022-01-05 23:53:59.000}{-14,-4,6}{10,10,10}{1.000}{20000000.000}{40000000.000}{3}{120}{40}{127}{12.572}
\apresdoc{2022-01-06\_142359.dat}{Grounding Line Camp}{Additional attenuator 10dB Rx, 0dB Tx.  Cable length Tx 25m, Rx 25m and separate antennas to maximum length (40m).  Antennas broadside.  Reconfigured to be towed in broadside configruation.  Basal return clear.}{2022-01-06 14:24:52.000}{-14,-4,6}{10,10,10}{1.000}{20000000.000}{40000000.000}{3}{120}{40}{127}{12.613}
\apresdoc{2022-01-06\_154601.dat}{Grounding Line Camp}{Additional attenuator 10dB Rx, 0dB Tx.  Cable length Tx 25m, Rx 25m and separate antennas to maximum length (40m).  Antennas broadside.  Reconfigured to be towed in broadside configruation.  Basal return clear.}{2022-01-06 15:47:50.000}{-14,-4,6}{10,10,10}{1.000}{20000000.000}{40000000.000}{3}{120}{40}{127}{12.431}
\apresdoc{2022-01-06\_155137.dat}{Grounding Line Camp}{Additional attenuator 10dB Rx, 0dB Tx.  Cable length Tx 25m, Rx 25m and separate antennas to maximum length (40m).  Antennas broadside.  Reconfigured to be towed in broadside configruation.  Basal return clear.  Measurement with high number of subbursts.}{2022-01-06 15:54:01.000}{-4}{10}{1.000}{20000000.000}{40000000.000}{1}{120}{120}{127}{12.423}
\apresdoc{2022-01-06\_164102.dat}{Grounding Line Camp}{Additional attenuator 10dB Rx, 0dB Tx.  Cable length Tx 25m, Rx 25m and separate antennas to maximum length (40m).  Antennas broadside.  Reconfigured to be towed in broadside configruation.  Tx connector split and replaced.  Tx/Rx distance reduced and slack added between Tx/Rx.  Basal return clear.}{2022-01-06 16:41:41.000}{-14,-4,6}{10,10,10}{1.000}{20000000.000}{40000000.000}{3}{120}{40}{127}{12.347}
\apresdoc{2022-01-06\_164609.dat}{Grounding Line Camp}{Additional attenuator 10dB Rx, 0dB Tx.  Cable length Tx 25m, Rx 25m and separate antennas to maximum length (40m).  Antennas broadside.  Reconfigured to be towed in broadside configruation.  Moving survey of 80 subbursts covering approx 30m or so?}{2022-01-06 16:48:45.000}{-4}{10}{1.000}{20000000.000}{40000000.000}{1}{80}{80}{127}{12.363}
\apresdoc{2022-01-07\_171120.dat}{Bulge Location}{Additional attenuator 10dB Rx, 0dB Tx.  Cable length Tx 25m, Rx 25m and separate antennas to maximum length (40m).  Antennas broadside.}{2022-01-07 17:11:29.000}{6}{10}{1.000}{20000000.000}{40000000.000}{1}{10}{10}{127}{12.306}
\apresdoc{2022-01-07\_180900.dat}{Bulge Location}{Additional attenuator 10dB Rx, 0dB Tx.  Cable length Tx 25m, Rx 25m and separate antennas to maximum length (40m).  Antennas broadside.}{2022-01-07 18:09:21.000}{6}{10}{1.000}{20000000.000}{40000000.000}{1}{20}{20}{127}{12.423}
\apresdoc{2022-01-08\_173538.dat}{Bulge Location}{Additional attenuator 10dB Rx, 0dB Tx.  Cable length Tx 25m, Rx 25m and separate antennas to maximum length (40m).  Antennas broadside.}{2022-01-08 17:37:19.000}{6}{10}{1.000}{20000000.000}{40000000.000}{1}{10}{10}{127}{12.061}
\apresdoc{bursttest.dat}{Bulge Location}{Need to investigate this - looks like a read or write error?}{2022-01-08 18:19:46.000}{6}{10}{1.000}{20000000.000}{40000000.000}{1}{5}{5}{127}{11.980}
\apresdoc{testburst2.dat}{Bulge Location}{Attenuator 10dB Rx, 0dB Tx}{2022-01-08 18:22:17.000}{6}{10}{1.000}{20000000.000}{40000000.000}{1}{5}{5}{127}{12.065}


\end{document}