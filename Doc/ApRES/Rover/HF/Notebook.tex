% -----------------------------------------------------------------------------
% Created:      2022-01-27
% Description:  Write-up of JH notebook and file log
%
% Requires:     csvsimple 
% -----------------------------------------------------------------------------

\documentclass[a4paper]{article}


% csvsimple package is required to load text from CSV files into tabular form
\usepackage{csvsimple}
\usepackage{xstring}
\usepackage{longtable}
\usepackage[table]{xcolor}
\usepackage{hyperref}

% array package for column modifications (emphasis)
\usepackage{array}

\usepackage{tikz}

\newcommand\hfaprestable[9]{
    % Arguments
    %   1 - filename
    %   2 - number of sub bursts
    %   3 - number of attenuator settings
    %   4 - AF gain (list)
    %   5 - RF attenuator (list)
    %   6 - lower frequency
    %   7 - upper frequency
    %   8 - chirp period
    %   9 - location
    %  10 - comments
    \def\temphftfolder{#1}
    \def\temphftfilename{#2}%
    \def\temphftsubbursts{#3}%
    \def\temphftnoattn{#4}
    \def\temphftaf{#5}%
    \def\temphftrf{#6}%
    \def\temphftlowfreq{#7}%
    \def\temphftupfreq{#8}%
    \def\temphftchirp{#9}%
    \hfaprestablecont
}

\newcommand{\afrftable}[1]{& #1}
\newcommand{\hfaprestablemcol}[1]{%
    % \multicolumn{4}{p{9cm}}{#1}%
    {#1}
}
\newcommand{\hfaprestablecont}[3]{
    \def\temphftlocation{#1}%
    \def\temphftcomment{#2}%
    \begin{table}[h]
        \centering
        \renewcommand{\arraystretch}{1.5}%
        \begin{tabular}{>{\columncolor{gray!10}}l p{9cm}}%
            \hline%
            \textbf{Folder} & \hfaprestablemcol{\ttfamily\url{\temphftfolder}} \\ % filename
            \textbf{Filename} & \hfaprestablemcol{\ttfamily\temphftfilename} \\ % filename
            \hline%
            Subbursts & \hfaprestablemcol{\temphftsubbursts} \\ %
            Attenuator Settings & \hfaprestablemcol{\temphftnoattn} \\ %
            AF Gain (dB) & 
            % \begin{tabular}{l l l l}
                \StrSubstitute{\temphftaf}{,}{, } \\%
            % \end{tabular}
            RF Attenuation (dB) & \StrSubstitute{\temphftrf}{,}{, } \\%
            \hline %
            Bandwidth ($B$) & \hfaprestablemcol{\temphftlowfreq~MHz to \temphftupfreq~MHz} \\ %
            Chirp Period ($T$) & \hfaprestablemcol{\temphftchirp~s} \\ %
            \hline %
            Location & \hfaprestablemcol{\temphftlocation} \\ %
            \hline %
            Comments & \hfaprestablemcol{\temphftcomment} \\ %
            \hline %
        \end{tabular} %
        \caption{File summary for \texttt{\temphftfilename}.}
        \label{tab:#3}
    \end{table}
    \clearpage
}

% -----------------------------------------------------------------------------
% Define command for database defintion tables
\newcommand{\sqlspectable}[2]{
    \renewcommand{\arraystretch}{1.5}
    \rowcolors{2}{gray!10}{white}
    \begin{longtable}{l l >{\raggedright}p{3cm} >{\raggedright\arraybackslash}p{4cm}}
        \caption{#2} \\
        \hline
        \textbf{Fieldname} & \textbf{Datatype} & \textbf{Parameters} & \textbf{Description}\\
        \hline
        \endhead
        \hline
        \endfoot
        #1
    \end{longtable}
}

% -----------------------------------------------------------------------------
% Title
\title{\textbf{Notebook Summary for HF ApRES Experiments}}
\author{J.D.~Hawkins}

\begin{document}

    \maketitle
    
    \newpage

    \tableofcontents

    \newpage

    \section{Database Structure}

    \subsection{Table `\texttt{measurements}'}
    \sqlspectable{
        %
        measurement\_id & INTEGER & PRIMARY KEY & 
        Unique identifier for each file. \\
        %
        filename & TEXT & NOT NULL & 
        Filename without path. \\
        %
        path & TEXT & UNIQUE NOT NULL&
        Path to file in UNIX form, relative to top-level project root. \\
        %
        name & TEXT & - &
        The name associated with the measurement, used to group sets of 
        measurements together. \\
        %
        timestamp & TEXT & UNIQUE ASC NOT NULL &
        \texttt{YYYY-mm-dd HH:MM:SS.fff} formatted timestamp according to
        time and date the measurement was taken. \\
        %
        valid & INTEGER & NOT NULL DEFAULT 0 &
        Boolean indicator of whether file is valid. Assumes invalid by 
        default. \\
        %
        location & TEXT & - & 
        Measurement location name. \\
        % 
        comments & TEXT & - &
        Description and comments for measurement if relevant. \\
        % 
        latitude & REAL & - &
        Latitude of measurement location if known. \\
        %
        longitude & REAL & - & 
        Longitude of measurement location if known. \\
        %
        elevation & REAL & - & 
        Elevation of measurement location if known (referenced to WGS84). \\
    }
    {Specification for \texttt{measurements} table.}
    
    \newpage

    \subsection{Table `\texttt{apres\_metadata}'}
    The fields \texttt{id} and \texttt{burst\_id} make a unique pair to ensure
    that each burst within a \texttt{*.dat} file is only represented once.

    \sqlspectable{
        %
        id & INTEGER & PRIMARY KEY & 
        Unique identifier for metadata.  Distinct from \texttt{measurement\_id}
        in that each \texttt{*.dat} file can have multiple bursts.\\
        %
        burst\_id & INTEGER & NOT NULL &
        Identifies the burst within a \texttt{*.dat} the metadata represents.\\
        %
        measurement\_id & INTEGER & NOT NULL & 
        Identifies the file record where the metadata originates from. \\
        \hline
        \multicolumn{4}{c}{\textit{ApRES Specific Metadata}} \\
        \hline
        %
        timestamp & TEXT & NOT NULL & 
        \texttt{YYYY-mm-dd HH:MM:SS.fff} formatted timestamp as logged in
        \texttt{*.dat} file. \\
        %
        n\_attenuators & INTEGER & NOT NULL CHECK(\textgreater 0, \textless 5) & 
        Number of attenuator settings used. \\
        %
        n\_chirps & INTEGER & NOT NULL CHECK(\textgreater 0) & 
        Total number of individual chirps in file. \\
        %
        n\_subbursts & INTEGER & NOT NULL CHECK(\textgreater 0) &
        Number of sub-bursts (repeats) of the burst configuration. \\
        %
        period & REAL & NOT NULL CHECK(\textgreater 0) & 
        Chirp period in seconds. \\
        %
        f\_lower & REAL & NOT NULL CHECK($\ge$ 0) &
        Lower bound of chirp ramp in Hertz. \\
        %
        f\_upper & REAL & NOT NULL CHECK($\ge$ 0) & 
        Upper bound of chirp ramp in Hertz. \\
        %
        af\_gain & TEXT & NOT NULL &
        Comma separated values indicating AF gain settings. \\
        %
        rf\_attenuator & TEXT & NOT NULL & 
        Comma separated values indicating RF attenuator settings. \\
        %
        f\_sampling & REAL & NOT NULL CHECK(\textgreater 0) & 
        Sampling frequency. \\
        % 
        tx\_antenna & TEXT & NOT NULL &
        Transmit antenna selection in comma separated value format. \\
        % 
        rx\_antenna & TEXT & NOT NULL &
        Receive antenna selection in comma separated value format. \\
        % 
        power\_code & INTEGER & - &
        DDS output current power code, if available (dependent on firmware).\\
        %
        battery\_voltage & REAL & - &
        Battery voltage, if available. \\
        %
        temperature\_1 & REAL & - &
        Measured board temperature from sensor 1. \\
        %
        temperature\_2 & REAL & - & 
        Measured board temperature from sensor 2. \\
        %
        rmb\_issue & TEXT & - & 
        RMB issue number if available. \\
        % 
        vab\_issue & TEXT & - & 
        VAB issue number if available. \\
        %
        venom\_issue & TEXT & - &
        Venom issue number if available. \\
        %
        software\_issue & TEXT & - & 
        VAB firmware issue number if available. \\
    }{
        Specification for \texttt{apres\_metadata} table.
    }
    
    \newpage

    \subsection{Table `\texttt{data}'}
    \sqlspectable{
        data\_id & INTEGER & PRIMARY KEY & 
        Unique identifier for each data item stored in \texttt{/Proc}. \\
        %
        measurement\_id & INTEGER & NOT NULL &
        Linked identified to \texttt{measurements} table for source of
        data. \\
        % 
        filename & TEXT & NOT NULL &
        Filename for processed data item stored in \texttt{/Proc}. \\
        %
        path & TEXT & NOT NULL & 
        Path to file in UNIX form, relative to top-level project root. \\
        %
        timestamp&  TEXT & ASC NOT NULL &
        \texttt{YYYY-mm-dd HH:MM:SS.SSS} formatted timestamp corresponding
        to time at which data was processed. \\
        %
        processing\_steps & TEXT & - & 
        Description of processing steps used to produced data file. \\
        %
    }
    {Specification for \texttt{data} table.}
    

    \newpage

    \section{Testing}
    % \hfaprestable
% Folder
{/Raw/ApRES/Rover/HF/Testing}%
% Filename
{1\_SubZero\_\_164126.90\_T1HR1H.dat}%
% Subbursts
{5}{1}{30}{-14}{20.000}{40.000}{1.000}%
% Location
{ (0.0000000, 0.0000000)}%
% Comments
{}%
% Label
{}%

\hfaprestable
% Foldertested
{/Raw/ApRES/Rover/HF/Testing}%
% Filename
{1\_SubZero\_\_165816.20\_T1HR1H.dat}%
% Subbursts
{5}{1}{30}{-14}{20.000}{40.000}{1.000}%
% Location
{ (0.0000000, 0.0000000)}%
% Comments
{}%
% Label
{}%

\hfaprestable
% Folder
{/Raw/ApRES/Rover/HF/Testing}%
% Filename
{1\_SubZero\_\_180215.40\_T1HR1H.dat}%
% Subbursts
{5}{1}{30}{6}{20.000}{40.000}{0.100}%
% Location
{ (0.0000000, 0.0000000)}%
% Comments
{}%
% Label
{}%

\hfaprestable
% Folder
{/Raw/ApRES/Rover/HF/Testing}%
% Filename
{1\_SubZero\_\_181539.50\_T1HR1H.dat}%
% Subbursts
{300}{1}{30}{-14}{20.000}{40.000}{1.000}%
% Location
{ (0.0000000, 0.0000000)}%
% Comments
{}%
% Label
{}%

\hfaprestable
% Folder
{/Raw/ApRES/Rover/HF/Testing}%
% Filename
{1\_SubZero\_\_211150.30\_T1HR1H.dat}%
% Subbursts
{5}{1}{30}{-14}{20.000}{40.000}{1.000}%
% Location
{ (0.0000000, 0.0000000)}%
% Comments
{}%
% Label
{}%

\hfaprestable
% Folder
{/Raw/ApRES/Rover/HF/Testing}%
% Filename
{2021-12-27\_214306.dat}%
% Subbursts
{20}{4}{30,20,10, 0}{6,6,6,6}{20.000}{40.000}{1.000}%
% Location
{ (0.0000000, 0.0000000)}%
% Comments
{}%
% Label
{}%

\hfaprestable
% Folder
{/Raw/ApRES/Rover/HF/Testing}%
% Filename
{2021-12-28\_213612.dat}%
% Subbursts
{1}{4}{30,20,10, 0}{6,6,6,6}{20.000}{40.000}{1.000}%
% Location
{ (0.0000000, 0.0000000)}%
% Comments
{}%
% Label
{}%

\hfaprestable
% Folder
{/Raw/ApRES/Rover/HF/Testing}%
% Filename
{2021-12-28\_213633.dat}%
% Subbursts
{1}{4}{30,20,10, 0}{6,6,6,6}{20.000}{40.000}{1.000}%
% Location
{ (0.0000000, 0.0000000)}%
% Comments
{}%
% Label
{}%

\hfaprestable
% Folder
{/Raw/ApRES/Rover/HF/Testing}%
% Filename
{2021-12-28\_213647.dat}%
% Subbursts
{1}{4}{30,20,10, 0}{6,6,6,6}{20.000}{40.000}{1.000}%
% Location
{ (0.0000000, 0.0000000)}%
% Comments
{}%
% Label
{}%

\hfaprestable
% Folder
{/Raw/ApRES/Rover/HF/Testing}%
% Filename
{2021-12-28\_213710.dat}%
% Subbursts
{1}{4}{30,20,10, 0}{6,6,6,6}{20.000}{40.000}{1.000}%
% Location
{ (0.0000000, 0.0000000)}%
% Comments
{}%
% Label
{}%

\hfaprestable
% Folder
{/Raw/ApRES/Rover/HF/Testing}%
% Filename
{2021-12-28\_213752.dat}%
% Subbursts
{10}{4}{30,20,10, 0}{6,6,6,6}{20.000}{40.000}{1.000}%
% Location
{ (0.0000000, 0.0000000)}%
% Comments
{}%
% Label
{}%

\hfaprestable
% Folder
{/Raw/ApRES/Rover/HF/Testing}%
% Filename
{2021-12-28\_214326.dat}%
% Subbursts
{10}{4}{30,20,10, 0}{6,6,6,6}{20.000}{40.000}{1.000}%
% Location
{ (0.0000000, 0.0000000)}%
% Comments
{}%
% Label
{}%

\hfaprestable
% Folder
{/Raw/ApRES/Rover/HF/Testing}%
% Filename
{2021-12-28\_214449.dat}%
% Subbursts
{10}{1}{30}{6}{20.000}{40.000}{1.000}%
% Location
{ (0.0000000, 0.0000000)}%
% Comments
{}%
% Label
{}%

\hfaprestable
% Folder
{/Raw/ApRES/Rover/HF/Testing}%
% Filename
{2021-12-28\_214537.dat}%
% Subbursts
{10}{1}{30}{-4}{20.000}{40.000}{1.000}%
% Location
{ (0.0000000, 0.0000000)}%
% Comments
{}%
% Label
{}%

\hfaprestable
% Folder
{/Raw/ApRES/Rover/HF/Testing}%
% Filename
{2021-12-28\_214702.dat}%
% Subbursts
{10}{1}{30}{-4}{20.000}{40.000}{1.000}%
% Location
{ (0.0000000, 0.0000000)}%
% Comments
{}%
% Label
{}%

\hfaprestable
% Folder
{/Raw/ApRES/Rover/HF/Testing}%
% Filename
{2021-12-28\_220421.dat}%
% Subbursts
{10}{1}{1}{-4}{20.000}{40.000}{1.000}%
% Location
{ (0.0000000, 0.0000000)}%
% Comments
{}%
% Label
{}%

\hfaprestable
% Folder
{/Raw/ApRES/Rover/HF/Testing}%
% Filename
{2021-12-28\_220508.dat}%
% Subbursts
{10}{1}{30}{-4}{20.000}{40.000}{1.000}%
% Location
{ (0.0000000, 0.0000000)}%
% Comments
{}%
% Label
{}%

\hfaprestable
% Folder
{/Raw/ApRES/Rover/HF/Testing}%
% Filename
{2021-12-28\_220547.dat}%
% Subbursts
{10}{1}{30}{-14}{20.000}{40.000}{1.000}%
% Location
{ (0.0000000, 0.0000000)}%
% Comments
{}%
% Label
{}%

\hfaprestable
% Folder
{/Raw/ApRES/Rover/HF/Testing}%
% Filename
{2021-12-28\_220726.dat}%
% Subbursts
{10}{1}{30}{-14}{20.000}{40.000}{1.000}%
% Location
{ (0.0000000, 0.0000000)}%
% Comments
{}%
% Label
{}%

\hfaprestable
% Folder
{/Raw/ApRES/Rover/HF/Testing}%
% Filename
{2021-12-28\_221042.dat}%
% Subbursts
{1}{1}{30}{-14}{20.000}{40.000}{1.000}%
% Location
{ (0.0000000, 0.0000000)}%
% Comments
{}%
% Label
{}%

\hfaprestable
% Folder
{/Raw/ApRES/Rover/HF/Testing}%
% Filename
{2021-12-28\_221153.dat}%
% Subbursts
{10}{1}{0}{-14}{20.000}{40.000}{1.000}%
% Location
{ (0.0000000, 0.0000000)}%
% Comments
{}%
% Label
{}%

\hfaprestable
% Folder
{/Raw/ApRES/Rover/HF/Testing}%
% Filename
{2021-12-28\_221319.dat}%
% Subbursts
{10}{1}{0}{-14}{20.000}{40.000}{1.000}%
% Location
{ (0.0000000, 0.0000000)}%
% Comments
{}%
% Label
{}%

\hfaprestable
% Folder
{/Raw/ApRES/Rover/HF/Testing}%
% Filename
{2021-12-28\_221436.dat}%
% Subbursts
{10}{1}{0}{-14}{20.000}{40.000}{1.000}%
% Location
{ (0.0000000, 0.0000000)}%
% Comments
{}%
% Label
{}%

\hfaprestable
% Folder
{/Raw/ApRES/Rover/HF/Testing}%
% Filename
{2021-12-28\_221516.dat}%
% Subbursts
{10}{1}{30}{6}{20.000}{40.000}{1.000}%
% Location
{ (0.0000000, 0.0000000)}%
% Comments
{}%
% Label
{}%

\hfaprestable
% Folder
{/Raw/ApRES/Rover/HF/Testing}%
% Filename
{2021-12-28\_221651.dat}%
% Subbursts
{10}{1}{30}{6}{20.000}{40.000}{1.000}%
% Location
{ (0.0000000, 0.0000000)}%
% Comments
{}%
% Label
{}%

\hfaprestable
% Folder
{/Raw/ApRES/Rover/HF/Testing}%
% Filename
{2021-12-28\_221725.dat}%
% Subbursts
{10}{1}{30}{-4}{20.000}{40.000}{1.000}%
% Location
{ (0.0000000, 0.0000000)}%
% Comments
{}%
% Label
{}%

\hfaprestable
% Folder
{/Raw/ApRES/Rover/HF/Testing}%
% Filename
{2021-12-28\_221834.dat}%
% Subbursts
{10}{1}{30}{-4}{20.000}{40.000}{1.000}%
% Location
{ (0.0000000, 0.0000000)}%
% Comments
{}%
% Label
{}%

\hfaprestable
% Folder
{/Raw/ApRES/Rover/HF/Testing}%
% Filename
{2021-12-28\_221937.dat}%
% Subbursts
{1}{1}{30}{-14}{20.000}{40.000}{1.000}%
% Location
{ (0.0000000, 0.0000000)}%
% Comments
{}%
% Label
{}%

\hfaprestable
% Folder
{/Raw/ApRES/Rover/HF/Testing}%
% Filename
{2021-12-28\_221958.dat}%
% Subbursts
{1}{1}{0}{6}{20.000}{40.000}{1.000}%
% Location
{ (0.0000000, 0.0000000)}%
% Comments
{}%
% Label
{}%

\hfaprestable
% Folder
{/Raw/ApRES/Rover/HF/Testing}%
% Filename
{2021-12-28\_222040.dat}%
% Subbursts
{1}{1}{0}{6}{20.000}{40.000}{1.000}%
% Location
{ (0.0000000, 0.0000000)}%
% Comments
{}%
% Label
{}%

\hfaprestable
% Folder
{/Raw/ApRES/Rover/HF/Testing}%
% Filename
{2021-12-28\_222046.dat}%
% Subbursts
{1}{1}{0}{6}{20.000}{40.000}{1.000}%
% Location
{ (0.0000000, 0.0000000)}%
% Comments
{}%
% Label
{}%

\hfaprestable
% Folder
{/Raw/ApRES/Rover/HF/Testing}%
% Filename
{2021-12-28\_222051.dat}%
% Subbursts
{1}{1}{0}{6}{20.000}{40.000}{1.000}%
% Location
{ (0.0000000, 0.0000000)}%
% Comments
{}%
% Label
{}%

\hfaprestable
% Folder
{/Raw/ApRES/Rover/HF/Testing}%
% Filename
{2021-12-28\_223632.dat}%
% Subbursts
{10}{1}{30}{-4}{20.000}{40.000}{1.000}%
% Location
{ (0.0000000, 0.0000000)}%
% Comments
{}%
% Label
{}%

\hfaprestable
% Folder
{/Raw/ApRES/Rover/HF/Testing}%
% Filename
{2021-12-28\_223835.dat}%
% Subbursts
{10}{1}{30}{-4}{20.000}{40.000}{1.000}%
% Location
{ (0.0000000, 0.0000000)}%
% Comments
{}%
% Label
{}%

\hfaprestable
% Folder
{/Raw/ApRES/Rover/HF/Testing}%
% Filename
{2021-12-28\_224425.dat}%
% Subbursts
{10}{1}{30}{-4}{36.000}{40.000}{1.000}%
% Location
{ (0.0000000, 0.0000000)}%
% Comments
{}%
% Label
{}%

\hfaprestable
% Folder
{/Raw/ApRES/Rover/HF/Testing}%
% Filename
{2021-12-28\_224607.dat}%
% Subbursts
{30}{1}{30}{-4}{36.000}{40.000}{1.000}%
% Location
{ (0.0000000, 0.0000000)}%
% Comments
{}%
% Label
{}%

\hfaprestable
% Folder
{/Raw/ApRES/Rover/HF/Testing}%
% Filename
{2021-12-28\_225108.dat}%
% Subbursts
{30}{1}{30}{-4}{36.000}{40.000}{1.000}%
% Location
{ (0.0000000, 0.0000000)}%
% Comments
{}%
% Label
{}%

\hfaprestable
% Folder
{/Raw/ApRES/Rover/HF/Testing}%
% Filename
{2021-12-28\_233118.dat}%
% Subbursts
{10}{4}{30,20,10, 0}{-14,-14,-14,-14}{20.000}{40.000}{1.000}%
% Location
{ (0.0000000, 0.0000000)}%
% Comments
{}%
% Label
{}%

\hfaprestable
% Folder
{/Raw/ApRES/Rover/HF/Testing}%
% Filename
{2021-12-28\_233345.dat}%
% Subbursts
{1}{1}{30}{-14}{20.000}{40.000}{1.000}%
% Location
{ (0.0000000, 0.0000000)}%
% Comments
{}%
% Label
{}%

\hfaprestable
% Folder
{/Raw/ApRES/Rover/HF/Testing}%
% Filename
{2021-12-28\_233417.dat}%
% Subbursts
{1}{1}{30}{-4}{20.000}{40.000}{1.000}%
% Location
{ (0.0000000, 0.0000000)}%
% Comments
{}%
% Label
{}%

\hfaprestable
% Folder
{/Raw/ApRES/Rover/HF/Testing}%
% Filename
{2021-12-28\_233617.dat}%
% Subbursts
{1}{1}{30}{-4}{20.000}{40.000}{1.000}%
% Location
{ (0.0000000, 0.0000000)}%
% Comments
{}%
% Label
{}%

\hfaprestable
% Folder
{/Raw/ApRES/Rover/HF/Testing}%
% Filename
{2021-12-28\_233652.dat}%
% Subbursts
{1}{1}{30}{-4}{20.000}{40.000}{1.000}%
% Location
{ (0.0000000, 0.0000000)}%
% Comments
{}%
% Label
{}%

\hfaprestable
% Folder
{/Raw/ApRES/Rover/HF/Testing}%
% Filename
{2021-12-28\_233714.dat}%
% Subbursts
{1}{1}{0}{-4}{20.000}{40.000}{1.000}%
% Location
{ (0.0000000, 0.0000000)}%
% Comments
{}%
% Label
{}%

\hfaprestable
% Folder
{/Raw/ApRES/Rover/HF/Testing}%
% Filename
{2021-12-28\_233729.dat}%
% Subbursts
{1}{1}{10}{-4}{20.000}{40.000}{1.000}%
% Location
{ (0.0000000, 0.0000000)}%
% Comments
{}%
% Label
{}%

\hfaprestable
% Folder
{/Raw/ApRES/Rover/HF/Testing}%
% Filename
{2021-12-28\_233742.dat}%
% Subbursts
{1}{1}{20}{-4}{20.000}{40.000}{1.000}%
% Location
{ (0.0000000, 0.0000000)}%
% Comments
{}%
% Label
{}%

\hfaprestable
% Folder
{/Raw/ApRES/Rover/HF/Testing}%
% Filename
{2021-12-28\_233844.dat}%
% Subbursts
{1}{1}{30}{-4}{20.000}{40.000}{1.000}%
% Location
{ (0.0000000, 0.0000000)}%
% Comments
{}%
% Label
{}%

\hfaprestable
% Folder
{/Raw/ApRES/Rover/HF/Testing}%
% Filename
{2021-12-28\_233949.dat}%
% Subbursts
{1}{1}{30}{-4}{20.000}{40.000}{1.000}%
% Location
{ (0.0000000, 0.0000000)}%
% Comments
{}%
% Label
{}%

\hfaprestable
% Folder
{/Raw/ApRES/Rover/HF/Testing}%
% Filename
{2021-12-28\_234104.dat}%
% Subbursts
{1}{1}{30}{-4}{20.000}{40.000}{1.000}%
% Location
{ (0.0000000, 0.0000000)}%
% Comments
{}%
% Label
{}%

\hfaprestable
% Folder
{/Raw/ApRES/Rover/HF/Testing}%
% Filename
{2021-12-28\_234122.dat}%
% Subbursts
{1}{1}{30}{-4}{20.000}{40.000}{1.000}%
% Location
{ (0.0000000, 0.0000000)}%
% Comments
{}%
% Label
{}%

\hfaprestable
% Folder
{/Raw/ApRES/Rover/HF/Testing}%
% Filename
{2021-12-28\_234222.dat}%
% Subbursts
{1}{1}{30}{-4}{20.000}{40.000}{1.000}%
% Location
{ (0.0000000, 0.0000000)}%
% Comments
{}%
% Label
{}%

\hfaprestable
% Folder
{/Raw/ApRES/Rover/HF/Testing}%
% Filename
{2021-12-29\_130718.dat}%
% Subbursts
{1}{1}{30}{-14}{20.000}{40.000}{1.000}%
% Location
{ (0.0000000, 0.0000000)}%
% Comments
{}%
% Label
{}%

\hfaprestable
% Folder
{/Raw/ApRES/Rover/HF/Testing}%
% Filename
{2021-12-29\_130920.dat}%
% Subbursts
{1}{1}{30}{-4}{20.000}{40.000}{1.000}%
% Location
{ (0.0000000, 0.0000000)}%
% Comments
{}%
% Label
{}%

\hfaprestable
% Folder
{/Raw/ApRES/Rover/HF/Testing}%
% Filename
{2021-12-29\_131140.dat}%
% Subbursts
{1}{1}{30}{6}{20.000}{40.000}{1.000}%
% Location
{ (0.0000000, 0.0000000)}%
% Comments
{}%
% Label
{}%

\hfaprestable
% Folder
{/Raw/ApRES/Rover/HF/Testing}%
% Filename
{2021-12-29\_131353.dat}%
% Subbursts
{1}{1}{30}{6}{20.000}{40.000}{1.000}%
% Location
{ (0.0000000, 0.0000000)}%
% Comments
{}%
% Label
{}%

\hfaprestable
% Folder
{/Raw/ApRES/Rover/HF/Testing}%
% Filename
{2021-12-29\_131431.dat}%
% Subbursts
{1}{4}{30,20,10, 0}{6,6,6,6}{20.000}{40.000}{1.000}%
% Location
{ (0.0000000, 0.0000000)}%
% Comments
{}%
% Label
{}%

\hfaprestable
% Folder
{/Raw/ApRES/Rover/HF/Testing}%
% Filename
{2021-12-29\_132044.dat}%
% Subbursts
{1}{4}{30,20,10, 0}{6,6,6,6}{20.000}{40.000}{1.000}%
% Location
{ (0.0000000, 0.0000000)}%
% Comments
{}%
% Label
{}%

\hfaprestable
% Folder
{/Raw/ApRES/Rover/HF/Testing}%
% Filename
{2021-12-29\_132217.dat}%
% Subbursts
{1}{1}{30}{6}{20.000}{40.000}{1.000}%
% Location
{ (0.0000000, 0.0000000)}%
% Comments
{}%
% Label
{}%

\hfaprestable
% Folder
{/Raw/ApRES/Rover/HF/Testing}%
% Filename
{2021-12-29\_133008.dat}%
% Subbursts
{1}{1}{30}{6}{20.000}{40.000}{1.000}%
% Location
{ (0.0000000, 0.0000000)}%
% Comments
{}%
% Label
{}%

\hfaprestable
% Folder
{/Raw/ApRES/Rover/HF/Testing}%
% Filename
{2021-12-29\_152201.dat}%
% Subbursts
{1}{1}{30}{6}{20.000}{40.000}{1.000}%
% Location
{ (0.0000000, 0.0000000)}%
% Comments
{}%
% Label
{}%

\hfaprestable
% Folder
{/Raw/ApRES/Rover/HF/Testing}%
% Filename
{2021-12-29\_154120.dat}%
% Subbursts
{1}{1}{30}{6}{20.000}{40.000}{1.000}%
% Location
{ (0.0000000, 0.0000000)}%
% Comments
{}%
% Label
{}%

\hfaprestable
% Folder
{/Raw/ApRES/Rover/HF/Testing}%
% Filename
{2021-12-29\_160719.dat}%
% Subbursts
{10}{1}{30}{6}{20.000}{40.000}{1.000}%
% Location
{ (0.0000000, 0.0000000)}%
% Comments
{}%
% Label
{}%

\hfaprestable
% Folder
{/Raw/ApRES/Rover/HF/Testing}%
% Filename
{2021-12-29\_161210.dat}%
% Subbursts
{1}{4}{30,20,10, 0}{6,6,6,6}{20.000}{40.000}{1.000}%
% Location
{ (0.0000000, 0.0000000)}%
% Comments
{}%
% Label
{}%

\hfaprestable
% Folder
{/Raw/ApRES/Rover/HF/Testing}%
% Filename
{2021-12-29\_163501.dat}%
% Subbursts
{1}{4}{30,20,10, 0}{6,6,6,6}{20.000}{40.000}{0.010}%
% Location
{ (0.0000000, 0.0000000)}%
% Comments
{}%
% Label
{}%

\hfaprestable
% Folder
{/Raw/ApRES/Rover/HF/Testing}%
% Filename
{2021-12-29\_164540.dat}%
% Subbursts
{1}{4}{30,20,10, 0}{6,6,6,6}{20.000}{40.000}{2.000}%
% Location
{ (0.0000000, 0.0000000)}%
% Comments
{}%
% Label
{}%

\hfaprestable
% Folder
{/Raw/ApRES/Rover/HF/Testing}%
% Filename
{2021-12-29\_165611.dat}%
% Subbursts
{1}{4}{30,20,10, 0}{6,6,6,6}{20.000}{40.000}{2.000}%
% Location
{ (0.0000000, 0.0000000)}%
% Comments
{}%
% Label
{}%

\hfaprestable
% Folder
{/Raw/ApRES/Rover/HF/Testing}%
% Filename
{2021-12-29\_170625.dat}%
% Subbursts
{1}{4}{30,20,10, 0}{6,6,6,6}{20.000}{40.000}{2.000}%
% Location
{ (0.0000000, 0.0000000)}%
% Comments
{}%
% Label
{}%

\hfaprestable
% Folder
{/Raw/ApRES/Rover/HF/Testing}%
% Filename
{2021-12-29\_171503.dat}%
% Subbursts
{1}{4}{30,20,10, 0}{6,6,6,6}{20.000}{40.000}{2.000}%
% Location
{ (0.0000000, 0.0000000)}%
% Comments
{}%
% Label
{}%

\hfaprestable
% Folder
{/Raw/ApRES/Rover/HF/Testing}%
% Filename
{2021-12-29\_172626.dat}%
% Subbursts
{1}{4}{30,20,10, 0}{6,6,6,6}{20.000}{40.000}{2.000}%
% Location
{ (0.0000000, 0.0000000)}%
% Comments
{}%
% Label
{}%

\hfaprestable
% Folder
{/Raw/ApRES/Rover/HF/Testing}%
% Filename
{2021-12-29\_173748.dat}%
% Subbursts
{1}{4}{30,20,10, 0}{6,6,6,6}{20.000}{40.000}{2.000}%
% Location
{ (0.0000000, 0.0000000)}%
% Comments
{}%
% Label
{}%

\hfaprestable
% Folder
{/Raw/ApRES/Rover/HF/Testing}%
% Filename
{2021-12-29\_180059.dat}%
% Subbursts
{1}{4}{30,20,10, 0}{6,6,6,6}{20.000}{40.000}{2.000}%
% Location
{ (0.0000000, 0.0000000)}%
% Comments
{}%
% Label
{}%

\hfaprestable
% Folder
{/Raw/ApRES/Rover/HF/Testing}%
% Filename
{2021-12-29\_180743.dat}%
% Subbursts
{1}{4}{30,20,10, 0}{6,6,6,6}{20.000}{40.000}{2.000}%
% Location
{ (0.0000000, 0.0000000)}%
% Comments
{}%
% Label
{}%

\hfaprestable
% Folder
{/Raw/ApRES/Rover/HF/Testing}%
% Filename
{2021-12-29\_181349.dat}%
% Subbursts
{1}{4}{30,20,10, 0}{6,6,6,6}{20.000}{40.000}{2.000}%
% Location
{ (0.0000000, 0.0000000)}%
% Comments
{}%
% Label
{}%

\hfaprestable
% Folder
{/Raw/ApRES/Rover/HF/Testing}%
% Filename
{2021-12-29\_181757.dat}%
% Subbursts
{1}{4}{30,20,10, 0}{6,6,6,6}{20.000}{40.000}{2.000}%
% Location
{ (0.0000000, 0.0000000)}%
% Comments
{}%
% Label
{}%

\hfaprestable
% Folder
{/Raw/ApRES/Rover/HF/Testing}%
% Filename
{2021-12-29\_182318.dat}%
% Subbursts
{1}{4}{30,20,10, 0}{6,6,6,6}{20.000}{40.000}{2.000}%
% Location
{ (0.0000000, 0.0000000)}%
% Comments
{}%
% Label
{}%

\hfaprestable
% Folder
{/Raw/ApRES/Rover/HF/Testing}%
% Filename
{2021-12-29\_183509.dat}%
% Subbursts
{1}{4}{30,20,10, 0}{6,6,6,6}{20.000}{40.000}{2.000}%
% Location
{ (0.0000000, 0.0000000)}%
% Comments
{}%
% Label
{}%

\hfaprestable
% Folder
{/Raw/ApRES/Rover/HF/Testing}%
% Filename
{2021-12-29\_184106.dat}%
% Subbursts
{1}{4}{30,20,10, 0}{6,6,6,6}{20.000}{40.000}{2.000}%
% Location
{ (0.0000000, 0.0000000)}%
% Comments
{}%
% Label
{}%

\hfaprestable
% Folder
{/Raw/ApRES/Rover/HF/Testing}%
% Filename
{2021-12-29\_184802.dat}%
% Subbursts
{1}{4}{30,20,10, 0}{6,6,6,6}{20.000}{40.000}{2.000}%
% Location
{ (0.0000000, 0.0000000)}%
% Comments
{}%
% Label
{}%

\hfaprestable
% Folder
{/Raw/ApRES/Rover/HF/Testing}%
% Filename
{2021-12-29\_185628.dat}%
% Subbursts
{1}{4}{30,20,10, 0}{6,6,6,6}{20.000}{40.000}{1.000}%
% Location
{ (0.0000000, 0.0000000)}%
% Comments
{}%
% Label
{}%

\hfaprestable
% Folder
{/Raw/ApRES/Rover/HF/Testing}%
% Filename
{2021-12-29\_190341.dat}%
% Subbursts
{1}{4}{30,20,10, 0}{6,6,6,6}{20.000}{40.000}{1.000}%
% Location
{ (0.0000000, 0.0000000)}%
% Comments
{}%
% Label
{}%

\hfaprestable
% Folder
{/Raw/ApRES/Rover/HF/Testing}%
% Filename
{2021-12-29\_190846.dat}%
% Subbursts
{1}{4}{30,20,10, 0}{6,6,6,6}{20.000}{40.000}{1.000}%
% Location
{ (0.0000000, 0.0000000)}%
% Comments
{}%
% Label
{}%

\hfaprestable
% Folder
{/Raw/ApRES/Rover/HF/Testing}%
% Filename
{2021-12-29\_191455.dat}%
% Subbursts
{1}{4}{30,20,10, 0}{6,6,6,6}{20.000}{40.000}{1.000}%
% Location
{ (0.0000000, 0.0000000)}%
% Comments
{}%
% Label
{}%

\hfaprestable
% Folder
{/Raw/ApRES/Rover/HF/Testing}%
% Filename
{2021-12-29\_191933.dat}%
% Subbursts
{1}{4}{30,20,10, 0}{6,6,6,6}{20.000}{40.000}{1.000}%
% Location
{ (0.0000000, 0.0000000)}%
% Comments
{}%
% Label
{}%

\hfaprestable
% Folder
{/Raw/ApRES/Rover/HF/Testing}%
% Filename
{2021-12-29\_192217.dat}%
% Subbursts
{1}{4}{30,20,10, 0}{6,6,6,6}{20.000}{40.000}{1.000}%
% Location
{ (0.0000000, 0.0000000)}%
% Comments
{}%
% Label
{}%

\hfaprestable
% Folder
{/Raw/ApRES/Rover/HF/Testing}%
% Filename
{2021-12-29\_193020.dat}%
% Subbursts
{1}{4}{30,20,10, 0}{6,6,6,6}{20.000}{40.000}{1.000}%
% Location
{ (0.0000000, 0.0000000)}%
% Comments
{}%
% Label
{}%

\hfaprestable
% Folder
{/Raw/ApRES/Rover/HF/Testing}%
% Filename
{2021-12-29\_193125.dat}%
% Subbursts
{1}{4}{30,20,10, 0}{6,6,6,6}{20.000}{40.000}{1.000}%
% Location
{ (0.0000000, 0.0000000)}%
% Comments
{}%
% Label
{}%

\hfaprestable
% Folder
{/Raw/ApRES/Rover/HF/Testing}%
% Filename
{2021-12-29\_193232.dat}%
% Subbursts
{1}{4}{30,20,10, 0}{6,6,6,6}{20.000}{40.000}{1.000}%
% Location
{ (0.0000000, 0.0000000)}%
% Comments
{}%
% Label
{}%

\hfaprestable
% Folder
{/Raw/ApRES/Rover/HF/Testing}%
% Filename
{2021-12-29\_193352.dat}%
% Subbursts
{1}{4}{30,20,10, 0}{6,6,6,6}{20.000}{40.000}{1.000}%
% Location
{ (0.0000000, 0.0000000)}%
% Comments
{}%
% Label
{}%

\hfaprestable
% Folder
{/Raw/ApRES/Rover/HF/Testing}%
% Filename
{2021-12-29\_193505.dat}%
% Subbursts
{1}{4}{30,20,10, 0}{6,6,6,6}{20.000}{40.000}{1.000}%
% Location
{ (0.0000000, 0.0000000)}%
% Comments
{}%
% Label
{}%

\hfaprestable
% Folder
{/Raw/ApRES/Rover/HF/Testing}%
% Filename
{2021-12-29\_214056.dat}%
% Subbursts
{1}{4}{30,20,10, 0}{6,6,6,6}{20.000}{40.000}{1.000}%
% Location
{ (0.0000000, 0.0000000)}%
% Comments
{}%
% Label
{}%

\hfaprestable
% Folder
{/Raw/ApRES/Rover/HF/Testing}%
% Filename
{2021-12-29\_214443.dat}%
% Subbursts
{1}{4}{30,20,10, 0}{6,6,6,6}{20.000}{40.000}{1.000}%
% Location
{ (0.0000000, 0.0000000)}%
% Comments
{}%
% Label
{}%

\hfaprestable
% Folder
{/Raw/ApRES/Rover/HF/Testing}%
% Filename
{2021-12-29\_215406.dat}%
% Subbursts
{1}{4}{30,20,10, 0}{6,6,6,6}{20.000}{40.000}{1.000}%
% Location
{ (0.0000000, 0.0000000)}%
% Comments
{}%
% Label
{}%

\hfaprestable
% Folder
{/Raw/ApRES/Rover/HF/Testing}%
% Filename
{2021-12-29\_215659.dat}%
% Subbursts
{1}{4}{30,20,10, 0}{6,6,6,6}{20.000}{40.000}{1.000}%
% Location
{ (0.0000000, 0.0000000)}%
% Comments
{}%
% Label
{}%

\hfaprestable
% Folder
{/Raw/ApRES/Rover/HF/Testing}%
% Filename
{2021-12-29\_220442.dat}%
% Subbursts
{1}{4}{30,20,10, 0}{6,6,6,6}{20.000}{40.000}{1.000}%
% Location
{ (0.0000000, 0.0000000)}%
% Comments
{}%
% Label
{}%

\hfaprestable
% Folder
{/Raw/ApRES/Rover/HF/Testing}%
% Filename
{2021-12-29\_220927.dat}%
% Subbursts
{1}{4}{30,20,10, 0}{6,6,6,6}{20.000}{40.000}{1.000}%
% Location
{ (0.0000000, 0.0000000)}%
% Comments
{}%
% Label
{}%

\hfaprestable
% Folder
{/Raw/ApRES/Rover/HF/Testing}%
% Filename
{2022-01-02\_182206.dat}%
% Subbursts
{5}{4}{30,20,10, 0}{6,6,6,6}{20.000}{40.000}{1.000}%
% Location
{ (0.0000000, 0.0000000)}%
% Comments
{}%
% Label
{}%

\hfaprestable
% Folder
{/Raw/ApRES/Rover/HF/Testing}%
% Filename
{2022-01-02\_182915.dat}%
% Subbursts
{10}{4}{30,20,10, 0}{6,6,6,6}{20.000}{40.000}{0.100}%
% Location
{ (0.0000000, 0.0000000)}%
% Comments
{}%
% Label
{}%

\hfaprestable
% Folder
{/Raw/ApRES/Rover/HF/Testing}%
% Filename
{2022-01-02\_183717.dat}%
% Subbursts
{10}{4}{30,20,10, 0}{6,6,6,6}{20.000}{40.000}{0.100}%
% Location
{ (0.0000000, 0.0000000)}%
% Comments
{}%
% Label
{}%

\hfaprestable
% Folder
{/Raw/ApRES/Rover/HF/Testing}%
% Filename
{2022-01-02\_185552.dat}%
% Subbursts
{1}{4}{30,20,10, 0}{6,6,6,6}{20.000}{40.000}{1.000}%
% Location
{ (0.0000000, 0.0000000)}%
% Comments
{}%
% Label
{}%

\hfaprestable
% Folder
{/Raw/ApRES/Rover/HF/Testing}%
% Filename
{2022-01-02\_185854.dat}%
% Subbursts
{1}{4}{30,20,10, 0}{6,6,6,6}{20.000}{40.000}{1.000}%
% Location
{ (0.0000000, 0.0000000)}%
% Comments
{}%
% Label
{}%

\hfaprestable
% Folder
{/Raw/ApRES/Rover/HF/Testing}%
% Filename
{2022-01-02\_191302.dat}%
% Subbursts
{1}{4}{30,20,10, 0}{6,6,6,6}{20.000}{40.000}{1.000}%
% Location
{ (0.0000000, 0.0000000)}%
% Comments
{}%
% Label
{}%

\hfaprestable
% Folder
{/Raw/ApRES/Rover/HF/Testing}%
% Filename
{2022-01-03\_115005.dat}%
% Subbursts
{20}{4}{30,20,10, 0}{6,6,6,6}{20.000}{40.000}{1.000}%
% Location
{ (0.0000000, 0.0000000)}%
% Comments
{}%
% Label
{}%

\hfaprestable
% Folder
{/Raw/ApRES/Rover/HF/Testing}%
% Filename
{2022-01-03\_144007.dat}%
% Subbursts
{1}{4}{30,20,10, 0}{6,6,6,6}{20.000}{40.000}{1.000}%
% Location
{ (0.0000000, 0.0000000)}%
% Comments
{}%
% Label
{}%

\hfaprestable
% Folder
{/Raw/ApRES/Rover/HF/Testing}%
% Filename
{2022-01-03\_152109.dat}%
% Subbursts
{10}{4}{30,20,10, 0}{6,6,6,6}{20.000}{40.000}{1.000}%
% Location
{ (0.0000000, 0.0000000)}%
% Comments
{}%
% Label
{}%

\hfaprestable
% Folder
{/Raw/ApRES/Rover/HF/Testing}%
% Filename
{2022-01-03\_153752.dat}%
% Subbursts
{10}{4}{30,20,10, 0}{6,6,6,6}{20.000}{40.000}{1.000}%
% Location
{ (0.0000000, 0.0000000)}%
% Comments
{}%
% Label
{}%

\hfaprestable
% Folder
{/Raw/ApRES/Rover/HF/Testing}%
% Filename
{2022-01-03\_161929.dat}%
% Subbursts
{10}{4}{30,20,10, 0}{6,6,6,6}{20.000}{40.000}{1.000}%
% Location
{ (0.0000000, 0.0000000)}%
% Comments
{}%
% Label
{}%

\hfaprestable
% Folder
{/Raw/ApRES/Rover/HF/Testing}%
% Filename
{2022-01-03\_164331.dat}%
% Subbursts
{1}{1}{10}{-14}{20.000}{40.000}{1.000}%
% Location
{ (0.0000000, 0.0000000)}%
% Comments
{}%
% Label
{}%

\hfaprestable
% Folder
{/Raw/ApRES/Rover/HF/Testing}%
% Filename
{2022-01-03\_170023.dat}%
% Subbursts
{40}{1}{30}{-14}{20.000}{40.000}{1.000}%
% Location
{ (0.0000000, 0.0000000)}%
% Comments
{}%
% Label
{}%

\hfaprestable
% Folder
{/Raw/ApRES/Rover/HF/Testing}%
% Filename
{2022-01-03\_173640.dat}%
% Subbursts
{20}{1}{30}{-14}{20.000}{40.000}{1.000}%
% Location
{ (0.0000000, 0.0000000)}%
% Comments
{}%
% Label
{}%

\hfaprestable
% Folder
{/Raw/ApRES/Rover/HF/Testing}%
% Filename
{2022-01-03\_174337.dat}%
% Subbursts
{20}{1}{30}{-14}{20.000}{40.000}{1.000}%
% Location
{ (0.0000000, 0.0000000)}%
% Comments
{}%
% Label
{}%

\hfaprestable
% Folder
{/Raw/ApRES/Rover/HF/Testing}%
% Filename
{2022-01-03\_174745.dat}%
% Subbursts
{20}{1}{30}{-14}{20.000}{40.000}{1.000}%
% Location
{ (0.0000000, 0.0000000)}%
% Comments
{}%
% Label
{}%

\hfaprestable
% Folder
{/Raw/ApRES/Rover/HF/Testing}%
% Filename
{2022-01-03\_175416.dat}%
% Subbursts
{40}{1}{30}{-14}{20.000}{40.000}{1.000}%
% Location
{ (0.0000000, 0.0000000)}%
% Comments
{}%
% Label
{}%

\hfaprestable
% Folder
{/Raw/ApRES/Rover/HF/Testing}%
% Filename
{2022-01-03\_180431.dat}%
% Subbursts
{40}{1}{30}{-14}{20.000}{40.000}{1.000}%
% Location
{ (0.0000000, 0.0000000)}%
% Comments
{}%
% Label
{}%

\hfaprestable
% Folder
{/Raw/ApRES/Rover/HF/Testing}%
% Filename
{2022-01-03\_192615.dat}%
% Subbursts
{40}{1}{30}{-14}{20.000}{40.000}{1.000}%
% Location
{ (0.0000000, 0.0000000)}%
% Comments
{}%
% Label
{}%

\hfaprestable
% Folder
{/Raw/ApRES/Rover/HF/Testing}%
% Filename
{2022-01-03\_194439.dat}%
% Subbursts
{60}{1}{30}{-4}{20.000}{40.000}{1.000}%
% Location
{ (0.0000000, 0.0000000)}%
% Comments
{}%
% Label
{}%

\hfaprestable
% Folder
{/Raw/ApRES/Rover/HF/Testing}%
% Filename
{2022-01-04\_110118.dat}%
% Subbursts
{40}{1}{30}{-14}{20.000}{40.000}{1.000}%
% Location
{ (0.0000000, 0.0000000)}%
% Comments
{}%
% Label
{}%

\hfaprestable
% Folder
{/Raw/ApRES/Rover/HF/Testing}%
% Filename
{2022-01-04\_133425.dat}%
% Subbursts
{40}{1}{30}{-14}{20.000}{40.000}{1.000}%
% Location
{ (0.0000000, 0.0000000)}%
% Comments
{}%
% Label
{}%

\hfaprestable
% Folder
{/Raw/ApRES/Rover/HF/Testing}%
% Filename
{2022-01-04\_133729.dat}%
% Subbursts
{40}{3}{30,20,10}{-14, -4,  6}{20.000}{40.000}{1.000}%
% Location
{ (0.0000000, 0.0000000)}%
% Comments
{}%
% Label
{}%

\hfaprestable
% Folder
{/Raw/ApRES/Rover/HF/Testing}%
% Filename
{2022-01-04\_142639.dat}%
% Subbursts
{1}{3}{30,30,30}{-14, -4,  6}{20.000}{40.000}{1.000}%
% Location
{ (0.0000000, 0.0000000)}%
% Comments
{}%
% Label
{}%

\hfaprestable
% Folder
{/Raw/ApRES/Rover/HF/Testing}%
% Filename
{2022-01-04\_142723.dat}%
% Subbursts
{40}{3}{30,30,30}{-14, -4,  6}{20.000}{40.000}{1.000}%
% Location
{ (0.0000000, 0.0000000)}%
% Comments
{}%
% Label
{}%

\hfaprestable
% Folder
{/Raw/ApRES/Rover/HF/Testing}%
% Filename
{2022-01-04\_153231.dat}%
% Subbursts
{40}{4}{30,20,30,20}{-14,-14, -4, -4}{20.000}{40.000}{1.000}%
% Location
{ (0.0000000, 0.0000000)}%
% Comments
{}%
% Label
{}%

\hfaprestable
% Folder
{/Raw/ApRES/Rover/HF/Testing}%
% Filename
{2022-01-05\_113341.dat}%
% Subbursts
{40}{2}{30,30}{ -4,-14}{20.000}{40.000}{1.000}%
% Location
{ (0.0000000, 0.0000000)}%
% Comments
{}%
% Label
{}%

\hfaprestable
% Folder
{/Raw/ApRES/Rover/HF/Testing}%
% Filename
{2022-01-05\_121151.dat}%
% Subbursts
{40}{2}{30,30}{ -4,-14}{20.000}{40.000}{0.100}%
% Location
{ (0.0000000, 0.0000000)}%
% Comments
{}%
% Label
{}%

\hfaprestable
% Folder
{/Raw/ApRES/Rover/HF/Testing}%
% Filename
{2022-01-05\_122713.dat}%
% Subbursts
{40}{2}{30,30}{-14, -4}{20.000}{40.000}{1.000}%
% Location
{ (0.0000000, 0.0000000)}%
% Comments
{}%
% Label
{}%

\hfaprestable
% Folder
{/Raw/ApRES/Rover/HF/Testing}%
% Filename
{2022-01-05\_123749.dat}%
% Subbursts
{40}{1}{30}{-14}{20.000}{40.000}{1.000}%
% Location
{ (0.0000000, 0.0000000)}%
% Comments
{}%
% Label
{}%

\hfaprestable
% Folder
{/Raw/ApRES/Rover/HF/Testing}%
% Filename
{2022-01-05\_124357.dat}%
% Subbursts
{40}{1}{30}{-14}{20.000}{40.000}{1.000}%
% Location
{ (0.0000000, 0.0000000)}%
% Comments
{}%
% Label
{}%

\hfaprestable
% Folder
{/Raw/ApRES/Rover/HF/Testing}%
% Filename
{2022-01-05\_140022.dat}%
% Subbursts
{40}{1}{30}{-14}{20.000}{40.000}{1.000}%
% Location
{ (0.0000000, 0.0000000)}%
% Comments
{}%
% Label
{}%

\hfaprestable
% Folder
{/Raw/ApRES/Rover/HF/Testing}%
% Filename
{2022-01-05\_141217.dat}%
% Subbursts
{40}{1}{30}{-14}{20.000}{40.000}{1.000}%
% Location
{ (0.0000000, 0.0000000)}%
% Comments
{}%
% Label
{}%

\hfaprestable
% Folder
{/Raw/ApRES/Rover/HF/Testing}%
% Filename
{2022-01-05\_142157.dat}%
% Subbursts
{8}{1}{30}{-14}{20.000}{40.000}{5.000}%
% Location
{ (0.0000000, 0.0000000)}%
% Comments
{}%
% Label
{}%

\hfaprestable
% Folder
{/Raw/ApRES/Rover/HF/Testing}%
% Filename
{2022-01-05\_142609.dat}%
% Subbursts
{1}{3}{30,30,30}{-14, -4,  6}{20.000}{40.000}{5.000}%
% Location
{ (0.0000000, 0.0000000)}%
% Comments
{}%
% Label
{}%

\hfaprestable
% Folder
{/Raw/ApRES/Rover/HF/Testing}%
% Filename
{2022-01-05\_142942.dat}%
% Subbursts
{1}{4}{30,20,10, 0}{-14,-14,-14,-14}{20.000}{40.000}{5.000}%
% Location
{ (0.0000000, 0.0000000)}%
% Comments
{}%
% Label
{}%

\hfaprestable
% Folder
{/Raw/ApRES/Rover/HF/Testing}%
% Filename
{2022-01-05\_145817.dat}%
% Subbursts
{40}{1}{30}{-14}{20.000}{40.000}{1.000}%
% Location
{ (0.0000000, 0.0000000)}%
% Comments
{}%
% Label
{}%

\hfaprestable
% Folder
{/Raw/ApRES/Rover/HF/Testing}%
% Filename
{2022-01-05\_150400.dat}%
% Subbursts
{40}{1}{30}{-14}{20.000}{40.000}{1.000}%
% Location
{ (0.0000000, 0.0000000)}%
% Comments
{}%
% Label
{}%

\hfaprestable
% Folder
{/Raw/ApRES/Rover/HF/Testing}%
% Filename
{2022-01-05\_152208.dat}%
% Subbursts
{40}{1}{30}{-14}{20.000}{40.000}{1.000}%
% Location
{ (0.0000000, 0.0000000)}%
% Comments
{}%
% Label
{}%

\hfaprestable
% Folder
{/Raw/ApRES/Rover/HF/Testing}%
% Filename
{2022-01-05\_153218.dat}%
% Subbursts
{40}{1}{30}{-14}{20.000}{40.000}{1.000}%
% Location
{ (0.0000000, 0.0000000)}%
% Comments
{}%
% Label
{}%

\hfaprestable
% Folder
{/Raw/ApRES/Rover/HF/Testing}%
% Filename
{2022-01-05\_155959.dat}%
% Subbursts
{40}{1}{30}{-14}{20.000}{40.000}{1.000}%
% Location
{ (0.0000000, 0.0000000)}%
% Comments
{}%
% Label
{}%

\hfaprestable
% Folder
{/Raw/ApRES/Rover/HF/Testing}%
% Filename
{2022-01-05\_163611.dat}%
% Subbursts
{40}{1}{30}{-14}{20.000}{40.000}{1.000}%
% Location
{ (0.0000000, 0.0000000)}%
% Comments
{}%
% Label
{}%

\hfaprestable
% Folder
{/Raw/ApRES/Rover/HF/Testing}%
% Filename
{2022-01-05\_164353.dat}%
% Subbursts
{40}{1}{0}{-14}{20.000}{40.000}{1.000}%
% Location
{ (0.0000000, 0.0000000)}%
% Comments
{}%
% Label
{}%

\hfaprestable
% Folder
{/Raw/ApRES/Rover/HF/Testing}%
% Filename
{2022-01-05\_164908.dat}%
% Subbursts
{40}{1}{0}{6}{20.000}{40.000}{1.000}%
% Location
{ (0.0000000, 0.0000000)}%
% Comments
{}%
% Label
{}%

\hfaprestable
% Folder
{/Raw/ApRES/Rover/HF/Testing}%
% Filename
{2022-01-05\_165236.dat}%
% Subbursts
{40}{1}{0}{-4}{20.000}{40.000}{1.000}%
% Location
{ (0.0000000, 0.0000000)}%
% Comments
{}%
% Label
{}%

\hfaprestable
% Folder
{/Raw/ApRES/Rover/HF/Testing}%
% Filename
{2022-01-05\_165821.dat}%
% Subbursts
{40}{3}{0,0,0}{-14, -4,  6}{20.000}{40.000}{1.000}%
% Location
{ (0.0000000, 0.0000000)}%
% Comments
{}%
% Label
{}%

\hfaprestable
% Folder
{/Raw/ApRES/Rover/HF/Testing}%
% Filename
{2022-01-05\_171029.dat}%
% Subbursts
{40}{3}{0,0,0}{-14, -4,  6}{20.000}{40.000}{1.000}%
% Location
{ (0.0000000, 0.0000000)}%
% Comments
{}%
% Label
{}%

\hfaprestable
% Folder
{/Raw/ApRES/Rover/HF/Testing}%
% Filename
{2022-01-05\_174143.dat}%
% Subbursts
{1}{3}{0,0,0}{-14, -4,  6}{20.000}{40.000}{1.000}%
% Location
{ (0.0000000, 0.0000000)}%
% Comments
{}%
% Label
{}%

\hfaprestable
% Folder
{/Raw/ApRES/Rover/HF/Testing}%
% Filename
{2022-01-05\_174254.dat}%
% Subbursts
{40}{3}{0,0,0}{-14, -4,  6}{20.000}{40.000}{1.000}%
% Location
{ (0.0000000, 0.0000000)}%
% Comments
{}%
% Label
{}%

\hfaprestable
% Folder
{/Raw/ApRES/Rover/HF/Testing}%
% Filename
{2022-01-05\_175045.dat}%
% Subbursts
{40}{3}{0,0,0}{-14, -4,  6}{20.000}{40.000}{1.000}%
% Location
{ (0.0000000, 0.0000000)}%
% Comments
{}%
% Label
{}%

\hfaprestable
% Folder
{/Raw/ApRES/Rover/HF/Testing}%
% Filename
{2022-01-05\_184909.dat}%
% Subbursts
{40}{3}{0,0,0}{-14, -4,  6}{20.000}{40.000}{1.000}%
% Location
{ (0.0000000, 0.0000000)}%
% Comments
{}%
% Label
{}%

\hfaprestable
% Folder
{/Raw/ApRES/Rover/HF/Testing}%
% Filename
{2022-01-05\_190323.dat}%
% Subbursts
{40}{3}{0,0,0}{-14, -4,  6}{20.000}{40.000}{1.000}%
% Location
{ (0.0000000, 0.0000000)}%
% Comments
{}%
% Label
{}%

\hfaprestable
% Folder
{/Raw/ApRES/Rover/HF/Testing}%
% Filename
{2022-01-05\_234755.dat}%
% Subbursts
{40}{3}{0,0,0}{-14, -4,  6}{20.000}{40.000}{1.000}%
% Location
{ (0.0000000, 0.0000000)}%
% Comments
{}%
% Label
{}%

\hfaprestable
% Folder
{/Raw/ApRES/Rover/HF/Testing}%
% Filename
{2022-01-05\_235346.dat}%
% Subbursts
{40}{3}{10,10,10}{-14, -4,  6}{20.000}{40.000}{1.000}%
% Location
{ (0.0000000, 0.0000000)}%
% Comments
{}%
% Label
{}%

\hfaprestable
% Folder
{/Raw/ApRES/Rover/HF/Testing}%
% Filename
{2022-01-06\_142359.dat}%
% Subbursts
{40}{3}{10,10,10}{-14, -4,  6}{20.000}{40.000}{1.000}%
% Location
{ (0.0000000, 0.0000000)}%
% Comments
{}%
% Label
{}%

\hfaprestable
% Folder
{/Raw/ApRES/Rover/HF/Testing}%
% Filename
{2022-01-06\_154601.dat}%
% Subbursts
{40}{3}{10,10,10}{-14, -4,  6}{20.000}{40.000}{1.000}%
% Location
{ (0.0000000, 0.0000000)}%
% Comments
{}%
% Label
{}%

\hfaprestable
% Folder
{/Raw/ApRES/Rover/HF/Testing}%
% Filename
{2022-01-06\_155137.dat}%
% Subbursts
{120}{1}{10}{-4}{20.000}{40.000}{1.000}%
% Location
{ (0.0000000, 0.0000000)}%
% Comments
{}%
% Label
{}%

\hfaprestable
% Folder
{/Raw/ApRES/Rover/HF/Testing}%
% Filename
{2022-01-06\_164102.dat}%
% Subbursts
{40}{3}{10,10,10}{-14, -4,  6}{20.000}{40.000}{1.000}%
% Location
{ (0.0000000, 0.0000000)}%
% Comments
{}%
% Label
{}%

\hfaprestable
% Folder
{/Raw/ApRES/Rover/HF/Testing}%
% Filename
{2022-01-06\_164609.dat}%
% Subbursts
{80}{1}{10}{-4}{20.000}{40.000}{1.000}%
% Location
{ (0.0000000, 0.0000000)}%
% Comments
{}%
% Label
{}%

\hfaprestable
% Folder
{/Raw/ApRES/Rover/HF/Testing}%
% Filename
{2022-01-07\_171120.dat}%
% Subbursts
{10}{1}{10}{6}{20.000}{40.000}{1.000}%
% Location
{ (0.0000000, 0.0000000)}%
% Comments
{}%
% Label
{}%

\hfaprestable
% Folder
{/Raw/ApRES/Rover/HF/Testing}%
% Filename
{2022-01-07\_180900.dat}%
% Subbursts
{20}{1}{10}{6}{20.000}{40.000}{1.000}%
% Location
{ (0.0000000, 0.0000000)}%
% Comments
{}%
% Label
{}%

\hfaprestable
% Folder
{/Raw/ApRES/Rover/HF/Testing}%
% Filename
{2022-01-08\_173538.dat}%
% Subbursts
{10}{1}{10}{6}{20.000}{40.000}{1.000}%
% Location
{ (0.0000000, 0.0000000)}%
% Comments
{}%
% Label
{}%

\hfaprestable
% Folder
{/Raw/ApRES/Rover/HF/Testing}%
% Filename
{2\_SubZero\_\_180505.20\_T1HR1H.dat}%
% Subbursts
{300}{1}{30}{6}{20.000}{40.000}{0.100}%
% Location
{ (0.0000000, 0.0000000)}%
% Comments
{}%
% Label
{}%

\hfaprestable
% Folder
{/Raw/ApRES/Rover/HF/Testing}%
% Filename
{DATA2022-01-03-1957.DAT}%
% Subbursts
{60}{1}{30}{-4}{20.000}{40.000}{1.000}%
% Location
{ (0.0000000, 0.0000000)}%
% Comments
{}%
% Label
{}%

\hfaprestable
% Folder
{/Raw/ApRES/Rover/HF/Testing}%
% Filename
{DATA2022-01-03-2002.DAT}%
% Subbursts
{60}{1}{30}{-4}{20.000}{40.000}{1.000}%
% Location
{ (0.0000000, 0.0000000)}%
% Comments
{}%
% Label
{}%

\hfaprestable
% Folder
{/Raw/ApRES/Rover/HF/Testing}%
% Filename
{DATA2022-01-03-2007.DAT}%
% Subbursts
{60}{1}{30}{-4}{20.000}{40.000}{1.000}%
% Location
{ (0.0000000, 0.0000000)}%
% Comments
{}%
% Label
{}%

\hfaprestable
% Folder
{/Raw/ApRES/Rover/HF/Testing}%
% Filename
{DATA2022-01-03-2012.DAT}%
% Subbursts
{60}{1}{30}{-4}{20.000}{40.000}{1.000}%
% Location
{ (0.0000000, 0.0000000)}%
% Comments
{}%
% Label
{}%

\hfaprestable
% Folder
{/Raw/ApRES/Rover/HF/Testing}%
% Filename
{DATA2022-01-03-2017.DAT}%
% Subbursts
{60}{1}{30}{-4}{20.000}{40.000}{1.000}%
% Location
{ (0.0000000, 0.0000000)}%
% Comments
{}%
% Label
{}%

\hfaprestable
% Folder
{/Raw/ApRES/Rover/HF/Testing}%
% Filename
{DATA2022-01-03-2022.DAT}%
% Subbursts
{60}{1}{30}{-4}{20.000}{40.000}{1.000}%
% Location
{ (0.0000000, 0.0000000)}%
% Comments
{}%
% Label
{}%

\hfaprestable
% Folder
{/Raw/ApRES/Rover/HF/Testing}%
% Filename
{DATA2022-01-03-2027.DAT}%
% Subbursts
{60}{1}{30}{-4}{20.000}{40.000}{1.000}%
% Location
{ (0.0000000, 0.0000000)}%
% Comments
{}%
% Label
{}%

\hfaprestable
% Folder
{/Raw/ApRES/Rover/HF/Testing}%
% Filename
{DATA2022-01-03-2032.DAT}%
% Subbursts
{60}{1}{30}{-4}{20.000}{40.000}{1.000}%
% Location
{ (0.0000000, 0.0000000)}%
% Comments
{}%
% Label
{}%

\hfaprestable
% Folder
{/Raw/ApRES/Rover/HF/Testing}%
% Filename
{DATA2022-01-03-2037.DAT}%
% Subbursts
{60}{1}{30}{-4}{20.000}{40.000}{1.000}%
% Location
{ (0.0000000, 0.0000000)}%
% Comments
{}%
% Label
{}%

\hfaprestable
% Folder
{/Raw/ApRES/Rover/HF/Testing}%
% Filename
{DATA2022-01-03-2042.DAT}%
% Subbursts
{60}{1}{30}{-4}{20.000}{40.000}{1.000}%
% Location
{ (0.0000000, 0.0000000)}%
% Comments
{}%
% Label
{}%

\hfaprestable
% Folder
{/Raw/ApRES/Rover/HF/Testing}%
% Filename
{DATA2022-01-03-2047.DAT}%
% Subbursts
{60}{1}{30}{-4}{20.000}{40.000}{1.000}%
% Location
{ (0.0000000, 0.0000000)}%
% Comments
{}%
% Label
{}%

\hfaprestable
% Folder
{/Raw/ApRES/Rover/HF/Testing}%
% Filename
{DATA2022-01-03-2052.DAT}%
% Subbursts
{60}{1}{30}{-4}{20.000}{40.000}{1.000}%
% Location
{ (0.0000000, 0.0000000)}%
% Comments
{}%
% Label
{}%

\hfaprestable
% Folder
{/Raw/ApRES/Rover/HF/Testing}%
% Filename
{DATA2022-01-03-2057.DAT}%
% Subbursts
{60}{1}{30}{-4}{20.000}{40.000}{1.000}%
% Location
{ (0.0000000, 0.0000000)}%
% Comments
{}%
% Label
{}%

\hfaprestable
% Folder
{/Raw/ApRES/Rover/HF/Testing}%
% Filename
{DATA2022-01-03-2102.DAT}%
% Subbursts
{60}{1}{30}{-4}{20.000}{40.000}{1.000}%
% Location
{ (0.0000000, 0.0000000)}%
% Comments
{}%
% Label
{}%

\hfaprestable
% Folder
{/Raw/ApRES/Rover/HF/Testing}%
% Filename
{DATA2022-01-03-2107.DAT}%
% Subbursts
{60}{1}{30}{-4}{20.000}{40.000}{1.000}%
% Location
{ (0.0000000, 0.0000000)}%
% Comments
{}%
% Label
{}%

\hfaprestable
% Folder
{/Raw/ApRES/Rover/HF/Testing}%
% Filename
{DATA2022-01-03-2112.DAT}%
% Subbursts
{60}{1}{30}{-4}{20.000}{40.000}{1.000}%
% Location
{ (0.0000000, 0.0000000)}%
% Comments
{}%
% Label
{}%

\hfaprestable
% Folder
{/Raw/ApRES/Rover/HF/Testing}%
% Filename
{DATA2022-01-03-2117.DAT}%
% Subbursts
{60}{1}{30}{-4}{20.000}{40.000}{1.000}%
% Location
{ (0.0000000, 0.0000000)}%
% Comments
{}%
% Label
{}%

\hfaprestable
% Folder
{/Raw/ApRES/Rover/HF/Testing}%
% Filename
{DATA2022-01-03-2122.DAT}%
% Subbursts
{60}{1}{30}{-4}{20.000}{40.000}{1.000}%
% Location
{ (0.0000000, 0.0000000)}%
% Comments
{}%
% Label
{}%

\hfaprestable
% Folder
{/Raw/ApRES/Rover/HF/Testing}%
% Filename
{DATA2022-01-04-0805.DAT}%
% Subbursts
{60}{1}{30}{-4}{20.000}{40.000}{1.000}%
% Location
{ (0.0000000, 0.0000000)}%
% Comments
{}%
% Label
{}%

\hfaprestable
% Folder
{/Raw/ApRES/Rover/HF/Testing}%
% Filename
{DATA2022-01-04-0810.DAT}%
% Subbursts
{60}{1}{30}{-4}{20.000}{40.000}{1.000}%
% Location
{ (0.0000000, 0.0000000)}%
% Comments
{}%
% Label
{}%

\hfaprestable
% Folder
{/Raw/ApRES/Rover/HF/Testing}%
% Filename
{DATA2022-01-04-0815.DAT}%
% Subbursts
{60}{1}{30}{-4}{20.000}{40.000}{1.000}%
% Location
{ (0.0000000, 0.0000000)}%
% Comments
{}%
% Label
{}%

\hfaprestable
% Folder
{/Raw/ApRES/Rover/HF/Testing}%
% Filename
{DATA2022-01-04-0820.DAT}%
% Subbursts
{60}{1}{30}{-4}{20.000}{40.000}{1.000}%
% Location
{ (0.0000000, 0.0000000)}%
% Comments
{}%
% Label
{}%

\hfaprestable
% Folder
{/Raw/ApRES/Rover/HF/Testing}%
% Filename
{DATA2022-01-04-0825.DAT}%
% Subbursts
{60}{1}{30}{-4}{20.000}{40.000}{1.000}%
% Location
{ (0.0000000, 0.0000000)}%
% Comments
{}%
% Label
{}%

\hfaprestable
% Folder
{/Raw/ApRES/Rover/HF/Testing}%
% Filename
{DATA2022-01-04-0830.DAT}%
% Subbursts
{60}{1}{30}{-4}{20.000}{40.000}{1.000}%
% Location
{ (0.0000000, 0.0000000)}%
% Comments
{}%
% Label
{}%

\hfaprestable
% Folder
{/Raw/ApRES/Rover/HF/Testing}%
% Filename
{DATA2022-01-04-0835.DAT}%
% Subbursts
{60}{1}{30}{-4}{20.000}{40.000}{1.000}%
% Location
{ (0.0000000, 0.0000000)}%
% Comments
{}%
% Label
{}%

\hfaprestable
% Folder
{/Raw/ApRES/Rover/HF/Testing}%
% Filename
{DATA2022-01-04-0840.DAT}%
% Subbursts
{60}{1}{30}{-4}{20.000}{40.000}{1.000}%
% Location
{ (0.0000000, 0.0000000)}%
% Comments
{}%
% Label
{}%

\hfaprestable
% Folder
{/Raw/ApRES/Rover/HF/Testing}%
% Filename
{DATA2022-01-04-0845.DAT}%
% Subbursts
{60}{1}{30}{-4}{20.000}{40.000}{1.000}%
% Location
{ (0.0000000, 0.0000000)}%
% Comments
{}%
% Label
{}%

\hfaprestable
% Folder
{/Raw/ApRES/Rover/HF/Testing}%
% Filename
{DATA2022-01-04-0850.DAT}%
% Subbursts
{60}{1}{30}{-4}{20.000}{40.000}{1.000}%
% Location
{ (0.0000000, 0.0000000)}%
% Comments
{}%
% Label
{}%

\hfaprestable
% Folder
{/Raw/ApRES/Rover/HF/Testing}%
% Filename
{DATA2022-01-04-0855.DAT}%
% Subbursts
{60}{1}{30}{-4}{20.000}{40.000}{1.000}%
% Location
{ (0.0000000, 0.0000000)}%
% Comments
{}%
% Label
{}%

\hfaprestable
% Folder
{/Raw/ApRES/Rover/HF/Testing}%
% Filename
{DATA2022-01-04-0900.DAT}%
% Subbursts
{60}{1}{30}{-4}{20.000}{40.000}{1.000}%
% Location
{ (0.0000000, 0.0000000)}%
% Comments
{}%
% Label
{}%

\hfaprestable
% Folder
{/Raw/ApRES/Rover/HF/Testing}%
% Filename
{DATA2022-01-04-0905.DAT}%
% Subbursts
{60}{1}{30}{-4}{20.000}{40.000}{1.000}%
% Location
{ (0.0000000, 0.0000000)}%
% Comments
{}%
% Label
{}%

\hfaprestable
% Folder
{/Raw/ApRES/Rover/HF/Testing}%
% Filename
{DATA2022-01-04-0910.DAT}%
% Subbursts
{60}{1}{30}{-4}{20.000}{40.000}{1.000}%
% Location
{ (0.0000000, 0.0000000)}%
% Comments
{}%
% Label
{}%

\hfaprestable
% Folder
{/Raw/ApRES/Rover/HF/Testing}%
% Filename
{DATA2022-01-04-0915.DAT}%
% Subbursts
{60}{1}{30}{-4}{20.000}{40.000}{1.000}%
% Location
{ (0.0000000, 0.0000000)}%
% Comments
{}%
% Label
{}%

\hfaprestable
% Folder
{/Raw/ApRES/Rover/HF/Testing}%
% Filename
{DATA2022-01-04-0920.DAT}%
% Subbursts
{60}{1}{30}{-4}{20.000}{40.000}{1.000}%
% Location
{ (0.0000000, 0.0000000)}%
% Comments
{}%
% Label
{}%

\hfaprestable
% Folder
{/Raw/ApRES/Rover/HF/Testing}%
% Filename
{Survey\_2022-01-02\_182206.dat}%
% Subbursts
{5}{4}{30,20,10, 0}{6,6,6,6}{20.000}{40.000}{1.000}%
% Location
{ (0.0000000, 0.0000000)}%
% Comments
{}%
% Label
{}%

\hfaprestable
% Folder
{/Raw/ApRES/Rover/HF/Testing}%
% Filename
{Survey\_2022-01-02\_182915.dat}%
% Subbursts
{10}{4}{30,20,10, 0}{6,6,6,6}{20.000}{40.000}{0.100}%
% Location
{ (0.0000000, 0.0000000)}%
% Comments
{}%
% Label
{}%

\hfaprestable
% Folder
{/Raw/ApRES/Rover/HF/Testing}%
% Filename
{Survey\_2022-01-02\_183717.dat}%
% Subbursts
{10}{4}{30,20,10, 0}{6,6,6,6}{20.000}{40.000}{0.100}%
% Location
{ (0.0000000, 0.0000000)}%
% Comments
{}%
% Label
{}%

\hfaprestable
% Folder
{/Raw/ApRES/Rover/HF/Testing}%
% Filename
{Survey\_2022-01-02\_185552.dat}%
% Subbursts
{1}{4}{30,20,10, 0}{6,6,6,6}{20.000}{40.000}{1.000}%
% Location
{ (0.0000000, 0.0000000)}%
% Comments
{}%
% Label
{}%

\hfaprestable
% Folder
{/Raw/ApRES/Rover/HF/Testing}%
% Filename
{Survey\_2022-01-02\_185854.dat}%
% Subbursts
{1}{4}{30,20,10, 0}{6,6,6,6}{20.000}{40.000}{1.000}%
% Location
{ (0.0000000, 0.0000000)}%
% Comments
{}%
% Label
{}%

\hfaprestable
% Folder
{/Raw/ApRES/Rover/HF/Testing}%
% Filename
{Survey\_2022-01-02\_191302.dat}%
% Subbursts
{1}{4}{30,20,10, 0}{6,6,6,6}{20.000}{40.000}{1.000}%
% Location
{ (0.0000000, 0.0000000)}%
% Comments
{}%
% Label
{}%

\hfaprestable
% Folder
{/Raw/ApRES/Rover/HF/Testing}%
% Filename
{Survey\_2022-01-03\_144007.dat}%
% Subbursts
{1}{4}{30,20,10, 0}{6,6,6,6}{20.000}{40.000}{1.000}%
% Location
{ (0.0000000, 0.0000000)}%
% Comments
{}%
% Label
{}%

\hfaprestable
% Folder
{/Raw/ApRES/Rover/HF/Testing}%
% Filename
{Survey\_2022-01-03\_152109.dat}%
% Subbursts
{10}{4}{30,20,10, 0}{6,6,6,6}{20.000}{40.000}{1.000}%
% Location
{ (0.0000000, 0.0000000)}%
% Comments
{}%
% Label
{}%

\hfaprestable
% Folder
{/Raw/ApRES/Rover/HF/Testing}%
% Filename
{Survey\_2022-01-03\_153752.dat}%
% Subbursts
{10}{4}{30,20,10, 0}{6,6,6,6}{20.000}{40.000}{1.000}%
% Location
{ (0.0000000, 0.0000000)}%
% Comments
{}%
% Label
{}%

\hfaprestable
% Folder
{/Raw/ApRES/Rover/HF/Testing}%
% Filename
{Survey\_2022-01-03\_161929.dat}%
% Subbursts
{10}{4}{30,20,10, 0}{6,6,6,6}{20.000}{40.000}{1.000}%
% Location
{ (0.0000000, 0.0000000)}%
% Comments
{}%
% Label
{}%

\hfaprestable
% Folder
{/Raw/ApRES/Rover/HF/Testing}%
% Filename
{Survey\_2022-01-04\_133425.dat}%
% Subbursts
{40}{1}{30}{-14}{20.000}{40.000}{1.000}%
% Location
{ (0.0000000, 0.0000000)}%
% Comments
{}%
% Label
{}%

\hfaprestable
% Folder
{/Raw/ApRES/Rover/HF/Testing}%
% Filename
{Survey\_2022-01-04\_133729.dat}%
% Subbursts
{40}{3}{30,20,10}{-14, -4,  6}{20.000}{40.000}{1.000}%
% Location
{ (0.0000000, 0.0000000)}%
% Comments
{}%
% Label
{}%

\hfaprestable
% Folder
{/Raw/ApRES/Rover/HF/Testing}%
% Filename
{Survey\_2022-01-04\_142639.dat}%
% Subbursts
{1}{3}{30,30,30}{-14, -4,  6}{20.000}{40.000}{1.000}%
% Location
{ (0.0000000, 0.0000000)}%
% Comments
{}%
% Label
{}%

\hfaprestable
% Folder
{/Raw/ApRES/Rover/HF/Testing}%
% Filename
{Survey\_2022-01-04\_142723.dat}%
% Subbursts
{40}{3}{30,30,30}{-14, -4,  6}{20.000}{40.000}{1.000}%
% Location
{ (0.0000000, 0.0000000)}%
% Comments
{}%
% Label
{}%

\hfaprestable
% Folder
{/Raw/ApRES/Rover/HF/Testing}%
% Filename
{Survey\_2022-01-04\_153231.dat}%
% Subbursts
{40}{4}{30,20,30,20}{-14,-14, -4, -4}{20.000}{40.000}{1.000}%
% Location
{ (0.0000000, 0.0000000)}%
% Comments
{}%
% Label
{}%

\hfaprestable
% Folder
{/Raw/ApRES/Rover/HF/Testing}%
% Filename
{Survey\_2022-01-05\_113341.dat}%
% Subbursts
{40}{2}{30,30}{ -4,-14}{20.000}{40.000}{1.000}%
% Location
{ (0.0000000, 0.0000000)}%
% Comments
{}%
% Label
{}%

\hfaprestable
% Folder
{/Raw/ApRES/Rover/HF/Testing}%
% Filename
{Survey\_2022-01-05\_121151.dat}%
% Subbursts
{40}{2}{30,30}{ -4,-14}{20.000}{40.000}{0.100}%
% Location
{ (0.0000000, 0.0000000)}%
% Comments
{}%
% Label
{}%

\hfaprestable
% Folder
{/Raw/ApRES/Rover/HF/Testing}%
% Filename
{Survey\_2022-01-05\_122713.dat}%
% Subbursts
{40}{2}{30,30}{-14, -4}{20.000}{40.000}{1.000}%
% Location
{ (0.0000000, 0.0000000)}%
% Comments
{}%
% Label
{}%

\hfaprestable
% Folder
{/Raw/ApRES/Rover/HF/Testing}%
% Filename
{Survey\_2022-01-05\_123749.dat}%
% Subbursts
{40}{1}{30}{-14}{20.000}{40.000}{1.000}%
% Location
{ (0.0000000, 0.0000000)}%
% Comments
{}%
% Label
{}%

\hfaprestable
% Folder
{/Raw/ApRES/Rover/HF/Testing}%
% Filename
{Survey\_2022-01-05\_124357.dat}%
% Subbursts
{40}{1}{30}{-14}{20.000}{40.000}{1.000}%
% Location
{ (0.0000000, 0.0000000)}%
% Comments
{}%
% Label
{}%

\hfaprestable
% Folder
{/Raw/ApRES/Rover/HF/Testing}%
% Filename
{Survey\_2022-01-05\_140022.dat}%
% Subbursts
{40}{1}{30}{-14}{20.000}{40.000}{1.000}%
% Location
{ (0.0000000, 0.0000000)}%
% Comments
{}%
% Label
{}%

\hfaprestable
% Folder
{/Raw/ApRES/Rover/HF/Testing}%
% Filename
{Survey\_2022-01-05\_141217.dat}%
% Subbursts
{40}{1}{30}{-14}{20.000}{40.000}{1.000}%
% Location
{ (0.0000000, 0.0000000)}%
% Comments
{}%
% Label
{}%

\hfaprestable
% Folder
{/Raw/ApRES/Rover/HF/Testing}%
% Filename
{Survey\_2022-01-05\_142157.dat}%
% Subbursts
{8}{1}{30}{-14}{20.000}{40.000}{5.000}%
% Location
{ (0.0000000, 0.0000000)}%
% Comments
{}%
% Label
{}%

\hfaprestable
% Folder
{/Raw/ApRES/Rover/HF/Testing}%
% Filename
{Survey\_2022-01-05\_142609.dat}%
% Subbursts
{1}{3}{30,30,30}{-14, -4,  6}{20.000}{40.000}{5.000}%
% Location
{ (0.0000000, 0.0000000)}%
% Comments
{}%
% Label
{}%

\hfaprestable
% Folder
{/Raw/ApRES/Rover/HF/Testing}%
% Filename
{Survey\_2022-01-05\_142942.dat}%
% Subbursts
{1}{4}{30,20,10, 0}{-14,-14,-14,-14}{20.000}{40.000}{5.000}%
% Location
{ (0.0000000, 0.0000000)}%
% Comments
{}%
% Label
{}%

\hfaprestable
% Folder
{/Raw/ApRES/Rover/HF/Testing}%
% Filename
{Survey\_2022-01-05\_145817.dat}%
% Subbursts
{40}{1}{30}{-14}{20.000}{40.000}{1.000}%
% Location
{ (0.0000000, 0.0000000)}%
% Comments
{}%
% Label
{}%

\hfaprestable
% Folder
{/Raw/ApRES/Rover/HF/Testing}%
% Filename
{Survey\_2022-01-05\_150400.dat}%
% Subbursts
{40}{1}{30}{-14}{20.000}{40.000}{1.000}%
% Location
{ (0.0000000, 0.0000000)}%
% Comments
{}%
% Label
{}%

\hfaprestable
% Folder
{/Raw/ApRES/Rover/HF/Testing}%
% Filename
{Survey\_2022-01-05\_152208.dat}%
% Subbursts
{40}{1}{30}{-14}{20.000}{40.000}{1.000}%
% Location
{ (0.0000000, 0.0000000)}%
% Comments
{}%
% Label
{}%

\hfaprestable
% Folder
{/Raw/ApRES/Rover/HF/Testing}%
% Filename
{Survey\_2022-01-05\_153218.dat}%
% Subbursts
{40}{1}{30}{-14}{20.000}{40.000}{1.000}%
% Location
{ (0.0000000, 0.0000000)}%
% Comments
{}%
% Label
{}%

\hfaprestable
% Folder
{/Raw/ApRES/Rover/HF/Testing}%
% Filename
{Survey\_2022-01-05\_155959.dat}%
% Subbursts
{40}{1}{30}{-14}{20.000}{40.000}{1.000}%
% Location
{ (0.0000000, 0.0000000)}%
% Comments
{}%
% Label
{}%

\hfaprestable
% Folder
{/Raw/ApRES/Rover/HF/Testing}%
% Filename
{Survey\_2022-01-05\_163611.dat}%
% Subbursts
{40}{1}{30}{-14}{20.000}{40.000}{1.000}%
% Location
{ (0.0000000, 0.0000000)}%
% Comments
{}%
% Label
{}%

\hfaprestable
% Folder
{/Raw/ApRES/Rover/HF/Testing}%
% Filename
{Survey\_2022-01-05\_164353.dat}%
% Subbursts
{40}{1}{0}{-14}{20.000}{40.000}{1.000}%
% Location
{ (0.0000000, 0.0000000)}%
% Comments
{}%
% Label
{}%

\hfaprestable
% Folder
{/Raw/ApRES/Rover/HF/Testing}%
% Filename
{Survey\_2022-01-05\_164908.dat}%
% Subbursts
{40}{1}{0}{6}{20.000}{40.000}{1.000}%
% Location
{ (0.0000000, 0.0000000)}%
% Comments
{}%
% Label
{}%

\hfaprestable
% Folder
{/Raw/ApRES/Rover/HF/Testing}%
% Filename
{Survey\_2022-01-05\_165236.dat}%
% Subbursts
{40}{1}{0}{-4}{20.000}{40.000}{1.000}%
% Location
{ (0.0000000, 0.0000000)}%
% Comments
{}%
% Label
{}%

\hfaprestable
% Folder
{/Raw/ApRES/Rover/HF/Testing}%
% Filename
{Survey\_2022-01-05\_165821.dat}%
% Subbursts
{40}{3}{0,0,0}{-14, -4,  6}{20.000}{40.000}{1.000}%
% Location
{ (0.0000000, 0.0000000)}%
% Comments
{}%
% Label
{}%

\hfaprestable
% Folder
{/Raw/ApRES/Rover/HF/Testing}%
% Filename
{Survey\_2022-01-05\_171029.dat}%
% Subbursts
{40}{3}{0,0,0}{-14, -4,  6}{20.000}{40.000}{1.000}%
% Location
{ (0.0000000, 0.0000000)}%
% Comments
{}%
% Label
{}%

\hfaprestable
% Folder
{/Raw/ApRES/Rover/HF/Testing}%
% Filename
{Survey\_2022-01-05\_174254.dat}%
% Subbursts
{40}{3}{0,0,0}{-14, -4,  6}{20.000}{40.000}{1.000}%
% Location
{ (0.0000000, 0.0000000)}%
% Comments
{}%
% Label
{}%

\hfaprestable
% Folder
{/Raw/ApRES/Rover/HF/Testing}%
% Filename
{Survey\_2022-01-05\_175045.dat}%
% Subbursts
{40}{3}{0,0,0}{-14, -4,  6}{20.000}{40.000}{1.000}%
% Location
{ (0.0000000, 0.0000000)}%
% Comments
{}%
% Label
{}%

\hfaprestable
% Folder
{/Raw/ApRES/Rover/HF/Testing}%
% Filename
{Survey\_2022-01-05\_184909.dat}%
% Subbursts
{40}{3}{0,0,0}{-14, -4,  6}{20.000}{40.000}{1.000}%
% Location
{ (0.0000000, 0.0000000)}%
% Comments
{}%
% Label
{}%

\hfaprestable
% Folder
{/Raw/ApRES/Rover/HF/Testing}%
% Filename
{Survey\_2022-01-05\_234755.dat}%
% Subbursts
{40}{3}{0,0,0}{-14, -4,  6}{20.000}{40.000}{1.000}%
% Location
{ (0.0000000, 0.0000000)}%
% Comments
{}%
% Label
{}%

\hfaprestable
% Folder
{/Raw/ApRES/Rover/HF/Testing}%
% Filename
{Survey\_2022-01-05\_235346.dat}%
% Subbursts
{40}{3}{10,10,10}{-14, -4,  6}{20.000}{40.000}{1.000}%
% Location
{ (0.0000000, 0.0000000)}%
% Comments
{}%
% Label
{}%

\hfaprestable
% Folder
{/Raw/ApRES/Rover/HF/Testing}%
% Filename
{Survey\_2022-01-06\_142359.dat}%
% Subbursts
{40}{3}{10,10,10}{-14, -4,  6}{20.000}{40.000}{1.000}%
% Location
{ (0.0000000, 0.0000000)}%
% Comments
{}%
% Label
{}%

\hfaprestable
% Folder
{/Raw/ApRES/Rover/HF/Testing}%
% Filename
{Survey\_2022-01-06\_154601.dat}%
% Subbursts
{40}{3}{10,10,10}{-14, -4,  6}{20.000}{40.000}{1.000}%
% Location
{ (0.0000000, 0.0000000)}%
% Comments
{}%
% Label
{}%

\hfaprestable
% Folder
{/Raw/ApRES/Rover/HF/Testing}%
% Filename
{Survey\_2022-01-06\_155137.dat}%
% Subbursts
{120}{1}{10}{-4}{20.000}{40.000}{1.000}%
% Location
{ (0.0000000, 0.0000000)}%
% Comments
{}%
% Label
{}%

\hfaprestable
% Folder
{/Raw/ApRES/Rover/HF/Testing}%
% Filename
{Survey\_2022-01-06\_164102.dat}%
% Subbursts
{40}{3}{10,10,10}{-14, -4,  6}{20.000}{40.000}{1.000}%
% Location
{ (0.0000000, 0.0000000)}%
% Comments
{}%
% Label
{}%

\hfaprestable
% Folder
{/Raw/ApRES/Rover/HF/Testing}%
% Filename
{Survey\_2022-01-06\_164609.dat}%
% Subbursts
{80}{1}{10}{-4}{20.000}{40.000}{1.000}%
% Location
{ (0.0000000, 0.0000000)}%
% Comments
{}%
% Label
{}%

\hfaprestable
% Folder
{/Raw/ApRES/Rover/HF/Testing}%
% Filename
{Survey\_2022-01-07\_171120.dat}%
% Subbursts
{10}{1}{10}{6}{20.000}{40.000}{1.000}%
% Location
{ (0.0000000, 0.0000000)}%
% Comments
{}%
% Label
{}%

\hfaprestable
% Folder
{/Raw/ApRES/Rover/HF/Testing}%
% Filename
{bursttest.dat}%
% Subbursts
{5}{1}{10}{6}{20.000}{40.000}{1.000}%
% Location
{ (0.0000000, 0.0000000)}%
% Comments
{}%
% Label
{}%

\hfaprestable
% Folder
{/Raw/ApRES/Rover/HF/Testing}%
% Filename
{hf-apres-test-2021-12-28.dat}%
% Subbursts
{1}{4}{30,20,10, 0}{6,6,6,6}{20.000}{40.000}{1.000}%
% Location
{ (0.0000000, 0.0000000)}%
% Comments
{}%
% Label
{}%

\hfaprestable
% Folder
{/Raw/ApRES/Rover/HF/Testing}%
% Filename
{testburst2.dat}%
% Subbursts
{5}{1}{10}{6}{20.000}{40.000}{1.000}%
% Location
{ (0.0000000, 0.0000000)}%
% Comments
{}%
% Label
{}%



\end{document}