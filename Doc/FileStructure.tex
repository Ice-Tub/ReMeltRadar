% -----------------------------------------------------------------------------
% Created:      2022-01-25
% Description:  Outline file structure for project directory
%
% Requires: 
% -----------------------------------------------------------------------------

\documentclass[a4paper]{article}

% Graphics packages
\usepackage{xcolor}
\usepackage[edges]{forest}

% -----------------------------------------------------------------------------
% Document Properties
\title{\textbf{File Structure for ReMeltRadar Project Repository}}
\author{R.~Drews, M.~R.~Ershadi, J.~D.~Hawkins, I.~Koch}

\begin{document}
\maketitle

% -----------------------------------------------------------------------------
% Front Matter
\linespread{1.5}

\section*{Description}
Outline of file structure for ReMeltRadar project repository to store data,
documentation and processing scripts.

\section*{Change Log}
\begin{table}[!h]
    \centering
        \begin{tabular}{l l p{8cm}}
        \hline
        Date & Version & Comment \\
        \hline
        2022-01-25 & 0.1 & Document created. \\
        \hline
    \end{tabular}
\end{table}
\newpage

% -----------------------------------------------------------------------------
% Overview of File Structure
\section{File Structure}
There are \textbf{two key file structures} within the project: the first is a
top-level file structure which discriminates between raw data, processed data
and  associated scripts and documentation.  The second is a sub-folder 
hierarchy which can be found within the top-level folders.  Top-level folders
which do not contain the sub-folder hierarchy are denoted with an asterisk
(\texttt{*}) in the figure below.

\subsection{Top-Level Folder Structure}
The purpose of each folder within the top-level folder structure  described 
in turn.

\begin{figure}[!h]
    \centering
    \begin{forest}
        for tree={folder,grow'=0,font=\ttfamily}
        [/ReMeltRadar
            [Doc]
            [QGis*]
            [Untouched]
            [Raw]
            [Proc]
            [Src]
            [Log*]
        ]
    \end{forest}
    \caption{Top-level folder structure for ReMeltRadar project repository.}
\end{figure}

\paragraph{Doc}
contains textual, imagery (i.e. scanned document or photography) or other 
documentation that describes how the data in \textbf{Raw} was collected
and how it is processed by scripts in \textbf{Src} to generate the output
data in \textbf{Proc}.

\paragraph{QGis}
contains the QGis project and associated shapefiles and imagery that were 
used for the ReMeltRadar traverse.

\paragraph{Untouched}
contains unprocessed data that has not been modified since export from the
instrument.  It serves as a backup of original data and \textbf{should not
be renamed, processed or otherwise modified}.

\paragraph{Raw}
contains unprocessed data that has been modified since originally downloaded.
This could include renaming of the files so that they are consistent, for
example.

\paragraph{Proc}
contains processed data that has been modified and processed using either
a script located in \textbf{Src} or otherwise.  The processing procedure
should be documented in \textbf{Doc} and the processed data should include
reference to the date on which they were generated.

\paragraph{Src}
contains scripts for processing the raw data located in \textbf{Raw} to 
generate the processed data destined for \textbf{Proc}.  Scripts 
\textbf{should not modified data in Raw} and instead modifications should
be made in-place within \textbf{Proc} if required.

\paragraph{Log}
is used to store log files and reports generated by scripts in \textbf{Src}
and through data cataloguing operations.

\subsection{Sub-Folder Structure}

The second file-structure can be found within most of the top-level folders 
and sorts data, processing scripts and documentation into a hierarchical 
structure.


\begin{figure}[!h]
    \centering
    \begin{forest}
        for tree={folder,grow'=0,font=\ttfamily}
        [/
            [ApRES
                [Rover
                    [HF]
                    [VHF]
                ]
                [SPM
                    [SPM\_A]
                    [SPM\_X]
                    [SPM\_X2]
                    [SPM\_X3]
                    [SPM\_X4]
                ]
            ]
            [CApRES]
            [PulseEKKO]
            [RTKGPS]
            [VNA
                [CableLength]
                [HFAntenna]    
            ]
        ]
    \end{forest}
    \caption{Overview of sub-folder file structure for hierarchical 
    separation of data, documentation and processing scripts.}
\end{figure}

% \begin{figure}
%     \dirtree{% 
%     .1 /.
%         .2 \textcolor{blue}{ApRES}.
%             .3 Rover.
%                 .4 HF.
%                 .4 VHF.
%             .3 SPM.
%                 .4 SPM\_A.
%                 .4 SPM\_X.
%                 .4 SPM\_X2.
%                 .4 SPM\_X3.
%                 .4 SPM\_X4.
%         .2 \textcolor{blue}{CApRES}.
%         .2 \textcolor{blue}{PulseEKKO}.
%         .2 \textcolor{blue}{RTKGPS}.
%         .2 \textcolor{blue}{VNA}.
%             .3 CableLength.
%             .3 HFAntenna.
%     }
%     \caption{Outline of sub-folder file structure to store data for each
%     measurement campaign}    
% \end{figure}


\end{document}